\documentclass{tufte-handout}

\usepackage{style}

\begin{document}

\begin{question}

Show that the integer \(n> 2\) with prime decomposistion \(n=p_1^{k_1}p_2^{k_2}...p_r^{k_r}\) 
is a square if, and only if, each of the exponents \(k_i\) is even.

\begin{align*}
n &= p_1^{2l_1}p_2^{2l_2}...p_r^{2l_r} \\
  &= (p_1^{l_1}p_2^{l_2}...p_r^{l_r})^2 \\
\stext{but}
n &= p_1^{2l_1+1}p_2^{2l_2+1}...p_r^{2l_r+1} \\
  &= (p_1^{1_1}p_2^{l_2}...p_r^{l_r})^2 \cdot p_1p_2...p_r \\
  \stext{as \( p_1 \neq p_2 \neq p_r \), hence not a square}
\end{align*}


\end{question}

%Question 4
\begin{question}

Show that \( p|hcf(a^2,b) \text{if and only if} p|(a,b^2) \)

\begin{align*}
    \stext{Using the definition of hcf;}
    \stext{Forward direction;}
p|hcf(a^2,b) \implies p|a^2 \text{ and } p|b\\[8pt]
\implies p|a \text{ and } p|b \\[8pt]
\implies p|hcf(a,b) \\[8pt]
\stext{Backward direction;}
p|hcf(a,b^2) \implies p|a \text{ and } p|b \\[8pt]
\implies p|a^2 \text{ and } p|b^2 \\[8pt]
\stext{Using Theorem 1.4 euclides lemma for prime numbers;}
\implies p|hcf(a^2,b^2) \\[8pt]
\implies p|hcf(a^2,b) \text{ and } p|hcf(a,b^2) \\[8pt]
\end{align*}

\end{question}

\begin{question}

Show that \( p \mid gcd(a^2, b) \) if and only if \( p \mid gcd(a, b^2) \).

\begin{proof}

Let \( p \) be a prime number.

\textbf{($\Rightarrow$)} Suppose \( p \mid \gcd(a^2, b) \).  
Then, by the definition of greatest common divisor,  
\[
p \mid a^2 \quad \text{and} \quad p \mid b.
\]
Since \( p \) is prime and \( p \mid a^2 \), by Euclid’s lemma we have \( p \mid a \).  
Hence,
\[
p \mid a \quad \text{and} \quad p \mid b,
\]
which implies \( p \mid \gcd(a, b^2) \) because \( p \mid b \Rightarrow p \mid b^2 \).

\medskip

\textbf{($\Leftarrow$)} Conversely, suppose \( p \mid \gcd(a, b^2) \).  
Then
\[
p \mid a \quad \text{and} \quad p \mid b^2.
\]
By Euclid’s lemma again, \( p \mid b^2 \Rightarrow p \mid b \).  
Therefore,
\[
p \mid a \quad \text{and} \quad p \mid b,
\]
which implies \( p \mid a^2 \) as well.  
Hence \( p \mid \gcd(a^2, b) \).

\medskip

Thus, we have shown both directions, and so
\[
p \mid \gcd(a^2, b) \iff p \mid \gcd(a, b^2).
\]

\end{proof}

\end{question}

\begin{question}

Show that \( p \mid hcf(a^2, b) \) if and only if \( p \mid hcf(a, b^2) \).

\begin{proof}

    \begin{align*}
\stext{Let \( p \) be a prime number.}
\stext{Forward direction (\(\Rightarrow\)):}
\stext{Assume \( p \mid hcf(a^2, b) \). } 
\stext{By the definition of highest common factor,}
p \mid a^2 \quad \text{and} \quad p \mid b.\\
\stext{Since \(p\) is prime, using Euclid’s lemma,}
p \mid a^2 \implies p \mid a.\\
\stext{Hence,}
p \mid a \quad \text{and} \quad p \mid b,\\
\stext{and therefore}
p \mid b \implies p \mid b^2.\\
\stext{Thus, \( p \mid hcf(a, b^2) \).}
\end{align*}


\begin{align*}
\stext{Backward direction (\(\Leftarrow\)):}
\stext{Assume \( p \mid hcf(a, b^2) \). } 
\stext{Then}
p \mid a \quad \text{and} \quad p \mid b^2.\\
\stext{Applying Euclid’s lemma again,}
p \mid b^2 \implies p \mid b.\\
\stext{Hence}
p \mid a \quad \text{and} \quad p \mid b,\\
\stext{and so}
p \mid a^2 \quad \text{and} \quad p \mid b.\\
\stext{Therefore, \( p \mid hcf(a^2, b) \).}
\stext{Since both directions hold, we conclude that}
p \mid hcf(a^2, b) \iff p \mid hcf(a, b^2)
\end{align*}

\end{proof}

\end{question}

%question 5
\begin{question}

If a nd b are positive integers and m and n are defined as

\[ m=3a+b \quad \text{ and } n=5a+2b \]

Then

\[ hcf(m,n) = hcf(a,b) \]

\begin{proof}
Let \( d = \gcd(a, b) \).  
Then we can write \( a = d a' \) and \( b = d b' \) for some integers \( a' \) 
and \( b' \) such that \( \gcd(a', b') = 1 \).

Now, substituting these into the expressions for \( m \) and \( n \):

\[ m = 3a + b = 3(da') + (db') = d(3a' + b') \]
\[ n = 5a + 2b = 5(da') + 2(db') = d(5a' + 2b') \]

Thus, both \( m \) and \( n \) are multiples of \( d \), which implies that \( d \) 
is a common divisor of \( m \) and \( n \).

Next, we need to show that a common divisor of \( m \) and \( n \) must also divide \( d \).

Let \( c \) be a common divisor of \( m \) and \( n \).  
Then \( c \mid m \) and \( c \mid n \), which means:

\[ c \mid d(3a' + b') \]
\[ c \mid d(5a' + 2b') \]   

\marginnote{Definition 4.6 \textit{Interger combination}}
Since \( c \) divides both \( d(3a' + b') \) and \( d(5a' + 2b') \), it must also divide any integer
linear combination of these two expressions. In particular, we can form the following combinations:

\[ 5m - 3n = 5d(3a' + b') - 3d(5a' + 2b') = d(15a' + 5b' - 15a' - 6b') = d(-b') \]
\[ 2n - m = 2d(5a' + 2b') - d(3a' + b') = d(10a' + 4b' - 3a' - b') = d(7a' + 3b') \]

\marginnote{Definition 4.8 \textit{Coprime}}
Since \( c \) divides both \( d(-b') \) and \( d(7a' + 3b') \), it follows that \( c \) must divide \( d \)
because \( a', b' \) are coprime, implies that \( c \) cannot divide \( a' \) or \( b' \) unless it divides \( d \).
Thus, we have shown that any common divisor \( c \) of \( m \) and \( n \) must also divide \( d \).
Therefore, the greatest common divisor of \( m \) and \( n \) is equal to \( d \):
\[ \gcd(m, n) = d = \gcd(a, b) \]
\end{proof}

\end{question}

\begin{question}

If \(a\) and \(b\) are positive integers and \(m, n\) are defined by
\[
m = 3a + b \quad \text{and} \quad n = 5a + 2b,
\]
show that
\[
\gcd(m, n) = \gcd(a, b).
\]

\begin{proof}

Let \(d = \gcd(a, b)\).  
Then \(a = da'\) and \(b = db'\) for some integers \(a', b'\) such that \(\gcd(a', b') = 1\).

\marginnote{Substitute \(a = da'\) and \(b = db'\)}
Substituting into the definitions of \(m\) and \(n\),
\[
m = 3a + b = 3(da') + (db') = d(3a' + b'),
\]
\[
n = 5a + 2b = 5(da') + 2(db') = d(5a' + 2b').
\]
Thus, \(d\) divides both \(m\) and \(n\), so \(d\) is a common divisor of \(m\) and \(n\).

\medskip

\marginnote{Any common divisor of \(m,n\) also divides \(d\)}
Let \(c\) be any common divisor of \(m\) and \(n\).  
Then
\[
c \mid d(3a' + b') \quad \text{and} \quad c \mid d(5a' + 2b').
\]

By the property of integer linear combinations, \(c\) must also divide any integer combination of these two:
\[
5m - 3n = 5d(3a' + b') - 3d(5a' + 2b') = d(-b'),
\]
\[
2n - m = 2d(5a' + 2b') - d(3a' + b') = d(7a' + 3b').
\]

\marginnote{Because \(\gcd(a', b') = 1\)}
Since \(a'\) and \(b'\) are coprime, the only common divisors of \((-b')\) and \((7a' + 3b')\) are \(\pm 1\).  
Hence, any divisor \(c\) of both expressions must divide \(d\).

\medskip

Thus, every common divisor of \(m, n\) divides \(d\), and \(d\) divides both \(m, n\).  
Therefore,
\[
\gcd(m, n) = d = \gcd(a, b).
\]

\end{proof}

\end{question}

%question 7

\begin{question}

Given a positive integer in the form \( 14a+3 \) there is a prime divisor of the form \( 14b+3 \).

\begin{proof}
Let \( n = 14a + 3 \) be a posistive integer.

If \( n \) is prime, then \( n \) must also be a prime divisor of the form \( 14b +3 \) for
some integer \( b \).

Hence;
\[ a = b \]

If \( n \) is not prime, then it must have a prime factor \( p \), such that \( p \mid n \).
By the properties of modular arithmetic, we have:

\[ n \equiv 3 \pmod 14 \]

Since \( p \mid n \), it follows that:

\[ p \equiv 3 \pmod 14 \]

Thus, \( p \) is a prime divisor of \( n \) of the form \( 14b + 3 \) for some integer \( b \).

\marginnote{Euclid's lemma}
Therefore, in either case, whether \( n \) is prime or not, there exists a prime divisor 
of \( n \) of the form \( 14b + 3 \).
\end{proof}

\end{question}

%question 8

\begin{question}

  If \( n \) is divisible by \( 15 \) but not by \( 9 \text{ or } 25 \), the \( \tau(n) \)
  is divisible by \( 4 \).

\begin{proof}

If \( n \) is divisble by 15, then its prime factorisation must be of the form;
\[ n = 3^{k_1} \cdot 5^{k_2} \cdot p_1^{k_3} \cdot p_2^{k_4} \ldots \]

\( k_1, k_2 = 1 \) as \( n \) is not divisble by \( 9 \text{ or } 25 \).

Hence we can write;
\[ n = 3^1 \cdot 5^1 \cdot p_1^{k_3} \cdot p_2^{k_4} \ldots \]
Where \( k_i > 1 \)

The number of divisors function \( \tau(n) \) is given by;
\[ \tau(n) = (1+1)(1+1)(k_3 +1)(k_4 + 1) \ldots \]
\[ \tau(n) = 4(k_3 +1)(k_4 + 1) \ldots \]
Thus, \( \tau(n) \) is divisible by \( 4 \).

\end{proof}

\end{question}

% Question 7 (corrected)
This one--check
\begin{question}

Given a positive integer in the form \( 14a + 3 \), determine whether there must exist 
a prime divisor of the form \( 14b + 3 \).

\begin{proof}
Let \( n = 14a + 3 \) for some integer \( a \). We consider two cases.

\marginnote{Case 1: \( n \) is prime}
If \( n \) is prime, then it is itself a prime divisor of the form \( 14b + 3 \) 
(taking \( b = a \)).

\marginnote{Case 2: \( n \) is composite}
If \( n \) is not prime, then it has at least one prime factor \( p \) such that \( p \mid n \).  
However, it does \emph{not} follow that \( p \equiv 3 \pmod{14} \) simply because 
\( n \equiv 3 \pmod{14} \). The congruence relation does not, in general, transfer 
from a number to all of its factors.

To see this, consider a counterexample:
\[
n = 185 = 14 \times 13 + 3.
\]
Then
\[
185 = 5 \times 37.
\]
Checking each prime factor modulo 14:
\[
5 \equiv 5 \pmod{14}, \quad 37 \equiv 9 \pmod{14}.
\]
Neither of these primes is congruent to \(3 \pmod{14}\).  
Therefore, \( n = 185 \) is a counterexample to the claim.

\marginnote{Conclusion}
Hence, the statement is \textbf{false}:  
not every integer of the form \( 14a + 3 \) has a prime divisor of the form \( 14b + 3 \).

\end{proof}

\end{question}

% Question 8 (verified true)

\begin{question}

If \( n \) is divisible by \( 15 \) but not by \( 9 \text{ or } 25 \), then \( \tau(n) \)
is divisible by \( 4 \).

\begin{proof}
If \( n \) is divisible by 15, its prime factorisation must be of the form
\[
n = 3^{k_1} \cdot 5^{k_2} \cdot p_1^{k_3} \cdot p_2^{k_4} \ldots
\]
Since \( n \) is not divisible by 9 or 25, we must have \( k_1 = k_2 = 1 \).

Hence
\[
n = 3^1 \cdot 5^1 \cdot p_1^{k_3} \cdot p_2^{k_4} \ldots
\]
where each \( k_i \geq 0 \).

The number of divisors function is given by
\[
\tau(n) = (k_1 + 1)(k_2 + 1)(k_3 + 1)(k_4 + 1)\ldots
\]
Substituting \( k_1 = k_2 = 1 \),
\[
\tau(n) = (1 + 1)(1 + 1)(k_3 + 1)(k_4 + 1)\ldots = 4 \times (k_3 + 1)(k_4 + 1)\ldots
\]
Since the remaining factors are integers, \( \tau(n) \) is a multiple of 4.

\marginnote{Conclusion}
Therefore, \( \tau(n) \) is divisible by 4 whenever \( n \) is divisible by 15 but not by 9 or 25.

\end{proof}

\end{question}

% Question 7 (corrected)

\begin{question}

Given a positive integer in the form \( 14a + 3 \), determine whether there must exist 
a prime divisor of the form \( 14b + 3 \).

\begin{proof}
Let \( n = 14a + 3 \) for some integer \( a \). We consider two cases.

\marginnote{Case 1: \( n \) is prime}
If \( n \) is prime, then it is itself a prime divisor of the form \( 14b + 3 \) 
(taking \( b = a \)).

\marginnote{Case 2: \( n \) is composite}
If \( n \) is not prime, then it has at least one prime factor \( p \) such that \( p \mid n \).  
However, it does \emph{not} follow that \( p \equiv 3 \pmod{14} \) simply because 
\( n \equiv 3 \pmod{14} \). The congruence relation does not, in general, transfer 
from a number to all of its factors.

To see this, consider a counterexample:
\[
n = 185 = 14 \times 13 + 3.
\]
Then
\[
185 = 5 \times 37.
\]
Checking each prime factor modulo 14:
\[
5 \equiv 5 \pmod{14}, \quad 37 \equiv 9 \pmod{14}.
\]
Neither of these primes is congruent to \(3 \pmod{14}\).  
Therefore, \( n = 185 \) is a counterexample to the claim.

\marginnote{Conclusion}
Hence, the statement is \textbf{false}:  
not every integer of the form \( 14a + 3 \) has a prime divisor of the form \( 14b + 3 \).

\end{proof}

\end{question}

% Question 8 (verified true)

\begin{question}

If \( n \) is divisible by \( 15 \) but not by \( 9 \text{ or } 25 \), then \( \tau(n) \)
is divisible by \( 4 \).

\begin{proof}
If \( n \) is divisible by 15, its prime factorisation must be of the form
\[
n = 3^{k_1} \cdot 5^{k_2} \cdot p_1^{k_3} \cdot p_2^{k_4} \ldots
\]
Since \( n \) is not divisible by 9 or 25, we must have \( k_1 = k_2 = 1 \).

Hence
\[
n = 3^1 \cdot 5^1 \cdot p_1^{k_3} \cdot p_2^{k_4} \ldots
\]
where each \( k_i \geq 0 \).

The number of divisors function is given by
\[
\tau(n) = (k_1 + 1)(k_2 + 1)(k_3 + 1)(k_4 + 1)\ldots
\]
Substituting \( k_1 = k_2 = 1 \),
\[
\tau(n) = (1 + 1)(1 + 1)(k_3 + 1)(k_4 + 1)\ldots = 4 \times (k_3 + 1)(k_4 + 1)\ldots
\]
Since the remaining factors are integers, \( \tau(n) \) is a multiple of 4.

\marginnote{Conclusion}
Therefore, \( \tau(n) \) is divisible by 4 whenever \( n \) is divisible by 15 but not by 9 or 25.

\end{proof}

\end{question}


\end{document}