\documentclass{tufte-handout}

\usepackage{style}

\begin{document}

\tma{01} % TMA number

%Question 1
\begin{question}

    \qpart
    \qsubpart

\functionmap[ALabel=A,BLabel=B,rowsep=4mm,colsep=30mm,padA=8pt,padB=10pt,color=myblue]
{1,2,3,4,5}
{Cat,Dog,Rabbit,Hamster,}
{1/1,2/4,3/2,4/3,5/1}

This is an example of a ONTO function because every element in the codomain has at least 
one arrow pointing to it from the domain. However, it is not a ONE-TO-ONE function because 
there are two arrows pointing to the same element in the codomain (1 and 5 both point to Cat).

\qsubpart

\functionmap[ALabel=A,BLabel=D,rowsep=4mm,colsep=30mm,padA=8pt,padB=10pt,color=myblue]
{1,2,3,4,5}
{\(\alpha\),\(\beta\),\(\gamma\),\(\delta\),}
{1/1,2/4,3/2,4/2,5/1}

This is neither a ONTO function nor a ONE-TO-ONE function. It is not ONTO because there is no
arrow pointing to \(\delta\) in the codomain. It is not ONE-TO-ONE because there are two arrows
pointing to the same element in the codomain.
    
\qsubpart

\functionmap[ALabel=A,BLabel=C,rowsep=4mm,colsep=30mm,padA=8pt,padB=10pt,color=myblue]
{1,2,3,4,5}
{\mathbb{Z}}
{1/1,2/1,3/1,4/1,5/1}

This is not a ONTO function because there are many elements in the codomain that do not have
an arrow pointing to them from the domain (e.g. 2, 3, -1, -2, etc.). However, it is a ONE-TO-ONE function
because no two arrows point to the same element in the codomain (all arrows point to \(\mathbb{Z}\)).

    \qsubpart

\functionmap[ALabel=B,BLabel=D,rowsep=4mm,colsep=30mm,padA=8pt,padB=10pt,color=myblue]
{Cat,Dog,Rabbit,Hamster,}
{\(\alpha\),\(\beta\),\(\gamma\),\(\delta\),}
{1/1,2/3,3/2,4/4}

This is both a ONTO function and a ONE-TO-ONE function, a bijection. It is ONTO because every element in the
codomain has at least one arrow pointing to it from the domain. It is ONE-TO-ONE because no two arrows
point to the same element in the codomain.

\qpart

\[ S = \{2,3,5,7\} \]
\[ T = \{1,3,5,7\} \]

\qsubpart

A one-one map of S to T.

\functionmap[ALabel=S,BLabel=T,rowsep=4mm,colsep=30mm,padA=8pt,padB=10pt,color=myblue]
{2,3,5,7}
{1,3,5,7}
{1/1,2/1,3/3,4/4}

\qsubpart

A bijection map of S to T.

\functionmap[ALabel=S,BLabel=T,rowsep=4mm,colsep=30mm,padA=8pt,padB=10pt,color=myblue]
{2,3,5,7}
{1,3,5,7}
{1/1,2/2,3/3,4/4}

\end{question}

%Question 2
\begin{question}

    \[ \text{for } n /geq 1, 7^{2n} - 6^{n} \text{ is divisible by } 43 \]

\begin{proof}
Using proof by induction we will first proove the basis of induction \( n = 1 \)

\begin{align*}
    P(1) &= 7^{2(1)} - 6^1\\[8pt]
    &= 49 - 6 \\[8pt]
    &= 43\\[8pt]
    \stext{Hence P(1) is divisible by 43}
\end{align*}

Now we will assume that \( P(k) \) is true for some arbitrary \( k \geq 1 \)
\begin{align*}
    P(k) &= 7^{2k} - 6^k \\[8pt]
    &= 43m\\[8pt]
\stext{ for some integer m}
\end{align*}

Next we will use this as our inductive hypothesis to prove that \( P(k+1) \) is true

\begin{align*}
P(k+1) &= 7^{2(k+1)} - 6^{k+1}\\[8pt]
&= 7^{2k+2} - 6^{k+1}\\[8pt]
&= 7^2 \cdot 7^{2k} - 6 \cdot 6^k\\[8pt]
&= 49 \cdot 7^{2k} - 6 \cdot 6^k\\[8pt]
\stext{From our inductive hypothesis we know that \( 7^{2k} - 6^k = 43m \)}\\[8pt]
&= 49(43m + 6^k) - 6 \cdot 6^k\\[8pt]
&= 49 \cdot 43m + 49 \cdot 6^k - 6 \cdot 6^k\\[8pt]
\stext{Factorising out the \( 6^k \)}
&= 49 \cdot 43m + (49 - 6) \cdot 6^k\\[8pt]
&= 49 \cdot 43m + 43 \cdot 6^k\\[8pt]
\stext{Factoring out the 43}\\[8pt]
&= 43(49m + 6^k)\\[8pt]
\stext{Since \( m \text{ and } k \) are both integers, \( 49m + 6^k \) is also an integer}\\[8pt]
\stext{hence \( P(k+1) \) is divisible by 43}\\[8pt]
\stext{By the principle of mathematical induction we have shown that \( P(n) \) is true for all \( n \geq 1 \)}
\end{align*}

\end{proof}

\end{question}

%Question 3
\begin{question}

\[ \gcd(2025,630) = 2025x +630y \]

\begin{align*}
2025 &= 3 \cdot \circled{630} + \circled{135}\\
\circled{630} &= 4 \cdot \circled{135} + \circled{90}\\
\circled{135} &= 1 \cdot \circled{90} + \circled{45}\\
\circled{90} &= 2 \cdot \circled{45} + 0\\
\stext{Hence, } \gcd(2025,630) = 45\\
\end{align*}

rearranging to make the remainder the subject;
\begin{align*}
45 &= 135 - 1 \cdot 90\\
90 &= 630 - 4 \cdot 135\\
135 &= 2025 - 3 \cdot 630\\
\end{align*}

Substituting back up the chain we get;
\begin{align*}
45 &= 135 - 1(630 - 4 \cdot 135)\\
&= 5 \cdot 135 - 1 \cdot 630\\
&= 5(2025 - 3 \cdot 630) - 1 \cdot 630\\
&= 5 \cdot 2025 - 15 \cdot 630 - 1 \cdot 630\\
&= 5 \cdot 2025 - 16 \cdot 630\\
\stext{Hence one solution is } 
x = 5, y = -16
\end{align*}

\end{question}

%Question 4
\begin{question}

\qpart

If \( a \) and \( b \) are positive integers and  \( p \) divides \( \gcd(a^2,b) \) if, and only if,
\( p \) divides \( \gcd(a,b^2) \).

\marginnote{(Definition 4.3) Factors and multiples: \( a \) divides \( b \) if there exists an integer \( q \) 
such that \( a = bq \).}
\marginnote{(Definition 4.5) Definition of gcd: The greatest common divisor of two integers \( a \) and \( b \), 
not both zero, is the largest positive integer that divides both \( a \) and \( b \).}  
\marginnote{(Theorem 1.4) Euclid's lemma for prime factors: If a prime \( p \) divides the product \( a_{1}a_{2} \ldots a_{n} \), 
then \( p \) must divide \( a_{i} \) for some integer \( i \), where \( a \leq i \leq n \).}

\begin{proof}

Let \( p \) be a prime number.

Forward direction (\( \Rightarrow \)):
Suppose \( p \mid \gcd(a^2,b) \) then by the definition 4.5,
\[ p \mid a^2 \text{ and } p \mid b \]

Since \( p \) is a prime nunber, and \( p \mid a^2 \), by Theorem 1.4,
\[ p \mid a \]
Hence,
\[ p \mid a \text{ and } p \mid b \]

Which implies, by definition 4.5,
\[ p \mid \gcd(a,b^2) \because p \mid b \implies p \mid b^2 \]

Backward direction (\( \Leftarrow \)):
conversely suppose \( p \mid \gcd(a,b^2) \) Then
\[ p \mid a \text{ and } p \mid b^2 \]

Again by Theorem 1.4, 
\[ p \mid b^2 \implies p \mid b \]
Therefore,
\[ p \mid a \text{ and } p \mid b \implies p \mid a^2 \text{ also } \]

Hence \( p \mid(a^2,b) \)

Thus, we have shown in both dirrections and so;
\[ p \mid \gcd(a^2,b) \Longleftrightarrow p \mid \gcd(a,b^2)  \]

\end{proof}

\vspace{5cm}

\qpart

If \( a \) and \( b \) are positive integers and \( m \) and \( n \) are
defined by 
\[ m = 3a + b \quad \text{ and } \quad n = 5a + 2b  \]
then \( \gcd(m,n) = \gcd(a,b) \)

\marginnote{(Theorem 4.4) Properties of division: If 
\( a \mid c \text{ and } a \mid d \) then \( a \mid (mc + nd) \) for any integers 
\( m text{ and } n \).}
\marginnote{(Definition 4.6) Integer combination: If \( a \text{ and } b \) are integers, then any 
integer of the form \( ma + nb \), with \( n,m \in \mathbb{Z} \), is called an integer combination
of \( a \text{ and } b \).}
\marginnote{(Definition 4.8) Coprime: Two integers \( a \text{ and } b \), notboth zero,
are coprime or relativly prime whenever \( \gcd(a,b) = 1 \).}


\begin{proof}

Let \( d = \gcd(a,b) \) and \( c = \gcd(m,n) \).

Since \( d \) is the greatest common divisor of \( a \) and \( b \), by definition 4.5,
\[ d \mid a \text{ and } d \mid b \]

Then by Theorem 4.4,
\[ d \mid (3a + b) \text{ and } d \mid (5a + 2b) \]
Hence,
\[ d \mid m \text{ and } d \mid n \]
Therefore, by definition 4.5,
\[ d \mid c \]  

And since \( c \) is the greatest common divisor of \( m \) and \( n \), by definition 4.5,
\[ c \mid m \text{ and } c \mid n \]
Then by Theorem 4.4,
\[ c \mid (2m - n) \text{ and } c \mid (5n - 3m) \]
Hence,
\[ c \mid a \text{ and } c \mid b \]
Therefore, by definition 4.5,
\[ c \mid d \]
Thus, \( d \mid c \text{ and } c \mid d \implies c = d \)
\[ \gcd(m,n) = \gcd(a,b) \]

\end{proof}

\vspace{5cm}

\qpart

A number of the form \( 14a + 3 \), where \( a \) is a non-negative integer, must have a prime
divisor of this same form \( 14b + 3 \), where \( b \) is a non-negative integer.

\marginnote{(Theorem 1.7) The fundamental theorem of arithmetic: Any integer \( n /geq 2 \) can be 
written uniquely in the form \( n = p_1^{k_1}p_2^{k_2} \ldots p_r^{k_r} \), where \( p_i \), 
\( i = 1, \ldots r \), are primes with \( p_1 < p_2 < \ldots p_r \) and each of \( k_i \),
\( i = 1, \ldots r \), is a natural number.}

\begin{proof}

    Consider if \( n \) is prime, then \( n \) is a prime divisor of itself and since
    \( n = 14a + 3 \), \( n \) is of the form \( 14b + 3 \) where \( b = a \).

    Now consider if \( n \) is composite, then by Theorem 1.7, \( n \) can be expressed as;
    \[ n = p_1^{k_1}p_2^{k_2} \ldots p_r^{k_r} \]
    thus having at leaset one prime factor such that \( p \mid n \)

    Hence we will have to show that;
    \[ n \equiv 3 \pmod{14} \]
        and
    \[ p \mid n, \text {such that } p \equiv 3 \pmod{14} \]

    Conside the counter example where \( n = 185 \), that is;
    \[ n = 185 = 5 \times 37 \]
    \[ 5 \equiv 5 \pmod{14} \]
    and
    \[ 37 \equiv 9 \pmod{14} \]
As niether of these prime factors are of the form \( 14b + 3 \) the stament is incorect.


\end{proof}

\vspace{5cm}

\qpart


\marginnote{(Definition 2.2) The \( \tau \) function: For any integer \(n \geq 1 \), \( \tau(n) \) is 
definied to be the number of distinct factors of \( n \), including \( 1 \text{ and } n \)}
\marginnote{(Proposition 2.3) The formula for \( \tau(n) \): For any integer \( n \geq 2 \), with
prime decomposition \( n = p_1^{k_1}p_2^{k_2} \ldots p_r^{k_r} \), we have \( \tau(n) = (k_1 + 1)(k_2 + 2) \ldots (k_r + 1) \). }

If \( n \) is divisible by \( 15 \) but not divisible by \( 9 \text{ or } 25 \), then \( \tau(n) \)
is divisible by \( 4 \).

\begin{proof}

By Theorem 1.7, we can express \( n \) as a product of its prime factors;
\[ n = 3^{k_1} \cdot 5^{k_2} \cdot p_1^{k_1} \cdot p_2^{k_2} \ldots \]
where \( k_1, k_2 = 1 \) as \( n \) is divisible by \( 15 \) but not by \( 9 \text{ or } 25 \).

Hence,
\[ n = 3^1 \cdot 5^1 \cdot p_1^{k_1} \cdot p_2^{k_2} \ldots \]

Then by Proposition 2.3, we can express \( \tau(n) \) as;
\begin{align*}
\tau(n) &= (1 + 1)(1 + 1)(k_1 + 1)(k_2 + 1) \ldots \\[8pt]
&= 2 \cdot 2 \cdot (k_1 + 1)(k_2 + 1) \ldots \\[8pt]
&= 4 \cdot (k_1 + 1)(k_2 + 1) \ldots \\[8pt]
\stext{Since \( (k_1 + 1)(k_2 + 1) \ldots \) is an integer, \( \tau(n) \) is divisible by 4}
\end{align*}

\end{proof}

\end{question}

%Question 5
\begin{question}

\qpart
\qsubpart

\marginnote{(Theorem 3.4) Solution of linear conguences}
\marginnote{(Theorem 4.2) Chinese Remainder Theorem}
\marginnote{(Theorem 4.5) Simutaneous solutions of linear congruences}

\begin{align*}
3x \equiv 4 \pmod{5}\\[8pt]
\stext{\( \gcd(3,5) = 1 \), hence \( 3 \) has a multiplicative inverse modulo \( 5 \)}\\
\stext{Hence, this will be the unique solution to the linear congruence}
3v \equiv 1 \pmod{5}\\[8pt]
\stext{By trying values for \( v \) we find that;}
3 \cdot 2 \equiv 1 \pmod{5}\\[8pt]
\stext{Thus, \( v = 2 \) is the multiplicative inverse of \( 3 \) modulo \( 5 \)}\\
\stext{Multiplying both sides of the original congruence by \( 2 \) gives;}
6x \equiv 8 \pmod{5}\\[8pt]
\stext{Reducing \( 6 \) and \( 8 \) modulo \( 5 \) gives;}
x \equiv 3 \pmod{5}\\[8pt]
\stext{Thus, the solution is;}
\boxed{x \equiv 3 \pmod{5}}
\end{align*}

\qsubpart

\begin{align*}
3x -1 \equiv 2(4 + x) \pmod{7}\\[8pt]
\stext{distributing the \( 2 \) on the RHS;}
3x - 1 \equiv 8 + 2x \pmod{7}\\[8pt]
\stext{subtracting \( 2x \) from both sides;}
x - 1 \equiv 8 \pmod{7}\\[8pt]
\stext{adding \( 1 \) to both sides;}
x \equiv 9 \pmod{7}\\[8pt]
\stext{reducing \( 9 \) modulo \( 7 \) gives;}
x \equiv 2 \pmod{7}\\[8pt]
\stext{Thus, the solution is;}
\boxed{x \equiv 2 \pmod{7}}
\end{align*}

\qsubpart

\begin{align*}
2(7 - x) \equiv 8 - x \pmod{17}\\[8pt]
\stext{distributing the \( 2 \) on the LHS;}
14 - 2x \equiv 8 - x \pmod{17}\\[8pt]
\stext{adding \( 2x \) to both sides;}
14 \equiv 8 + x \pmod{17}\\[8pt]
\stext{subtracting \( 8 \) from both sides;}
6 \equiv x \pmod{17}\\[8pt]
\stext{Thus, the solution is;}
\boxed{x \equiv 6 \pmod{17}}
\end{align*}

\qpart

\[ x \equiv 3 \pmod{5} \]
\[ x \equiv 2 \pmod{7}  \]
\[ x \equiv 6 \pmod{17} \]
\[ \gcd(5,2) = 1 \quad \gcd(7,17) = 1 \quad \gcd(7,17) = 1 \]

Hence by Theorem 4.5, the uniques solution is \( \lcm(3 \pmod{5}, 2 \pmod{7}, 6 \pmod{17}) \);


\[ 3, 8, 13, 18, \circled{23} \]
\[ 2, 9 16, \circled{23} \]
\[ 6, \circled{23} \]

\[ \boxed{x = 23 } \]

\end{question}

\end{document}