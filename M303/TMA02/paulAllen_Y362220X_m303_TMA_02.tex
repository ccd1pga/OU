\documentclass{tufte-handout}

\usepackage{style}

\begin{document}

\tma{02} % TMA number

\section{Part A}

%Question 1
\begin{question}

\end{question}


%Question 2
\begin{question}

\[ G = \{ a + b\sqrt{-5} : a,b \in \mathbb{Z} \} \]
and \( \boldsymbol{+}_{G} \) be
\[ (a + a^\prime) + (b + b^\prime)\sqrt{-5} \]
and a sunset of \( G \)
\[ H = \{ a + b\sqrt{-5} : a,b \in 4\mathbb{Z} \} \]

\vspace{1cm}

\qpart

\qsubpart

The identity element, \( e \), of \( G \) is, \( 0 + 0\sqrt{-5} \).

\qsubpart

An element of \( H \) that is not the identity is, \( 4 + 8\sqrt{-5} \).

\qsubpart

An element of \( G \) that is not in \( H \) is, \( 3 + 5\sqrt{-5} \).

\qsubpart

The sum, under \( \boldsymbol{+}_{G} \), of these two elements are,
\begin{align*}
(4 + 8\sqrt{-5}) + (3 + 5\sqrt{-5}) &= (4 + 3) + (8 + 5)\sqrt{-5}\\
&= 7 + 13\sqrt{-5}
\end{align*}

\vspace{5cm}

\qpart

\textbf{Closure}:

Let,
\[ x = a + b\sqrt{-5} \quad \text{ and } y = a^\prime + b^\prime\sqrt{-5} \]

then
\[ x + y = (a + a^\prime) + (b + b^\prime)\sqrt{-5} \]

and as \( \mathbb{Z} \) is closed under addition \( a + a^\prime \in \mathbb{Z} \text{ and } b + b^\prime \in \mathbb{Z}\),
Hence, \( x + y \in G \).

\vspace{2cm}

\textbf{Associativity}:

Let,
\[ x = a + b\sqrt{-5}, \quad y = a^\prime + b^\prime\sqrt{-5}, \quad z = a^{\prime\prime} + b^{\prime\prime}\sqrt{-5} \]

then
\begin{align*}
(x + y) + z &= ((a + a^\prime) + a^{\prime\prime}) + ((b + b^\prime) + b^{\prime\prime})\sqrt{-5}\\[8pt]
x + (y + z) &= (a + (a^\prime + a^{\prime\prime})) + (b + (b^\prime + b^{\prime\prime}))\sqrt{-5}\\[8pt]
\stext{As addition in \( \mathbb{Z} \) is associative, }
(a + a^\prime) + a^{\prime\prime} &= a + (a^\prime + a^{\prime\prime})\\
\stext{and}
(b + b^\prime) + b^{\prime\prime} &= b + (b^\prime + b^{\prime\prime})
\stext{Thus, }
(x + y) + z &= x + (y + z)
\end{align*}

\vspace{2cm}

\textbf{Identity element}:

Let,
\[ e = 0 + 0\sqrt{-5} \]
and
\[ x = a + b\sqrt{-5} \]
then
\begin{align*}
x + e &= (a + 0) + (b + 0)\sqrt{-5}\\
&= a + b\sqrt{-5}\\
&= x
\end{align*}

\vspace{2cm}

\textbf{Inverse element}:

Let,
\[ x = a + b\sqrt{-5} \]
then
\begin{align*}
x + (-x) &= (a + -a) + (b + -b)\sqrt{-5}\\
&= 0 + 0\sqrt{-5}\\
&= e
\end{align*}

Hence \( (G, \boldsymbol{+}_{G}) \) is a group.

\vspace{5cm}

\qpart

Using \textup{Proposition 2.5; Subgroup criteria, HB Chapter 5 p26.}

\[ x,y \in H \implies x - y \in H \]

Let,
\[ x = 4m + 4n\sqrt{-5} \quad \text{ and } y = 4p + 4q\sqrt{-5} \]

then
\begin{align*}
x - y &= (4m - 4p) + (4n - 4q)\sqrt{-5}\\
&= 4(m - p) + 4(n - q)\sqrt{-5}
\end{align*}

As \( m - p \in \mathbb{Z} \text{ and } n - q \in \mathbb{Z} \), then \( x - y \in H \).
Hence, \( H \) is a subgroup of \( G \).

\vspace{5cm}

\qpart

The quotient order of the group \( \frac{G}{H} \) is 16.

\vspace{2cm}

\qpart

The order of the element \( 1 +\sqrt{-5} + H \in \frac{G}{H} \) is 4.

\end{question}

%Question 3
\begin{question}

\[ M = \begin{pmatrix}
a,0\\
b,c
\end{pmatrix} : a,b,c \in \mathbb{Z}_5 \]

\qpart

The additive inverse is 
\[ \begin{pmatrix}
-a,-0\\
-b,-c
\end{pmatrix} \]

\qpart

Let 
\[ \varphi: M \rightarrow \mathbb{Z}_5\]
be the map that takes a matrix in \( M \) to the lower left entry in \( M \).

\begin{proof}
To show that \( \varphi \) is a homomorphism, let
\[ X = \begin{pmatrix}
a,0\\
b,c
\end{pmatrix} \quad \text{ and } \quad Y = \begin{pmatrix}
d,0\\
e,f
\end{pmatrix} \]
then
\begin{align*}
\varphi(X + Y) &= \varphi\begin{pmatrix}
a + d,0 + 0\\
b + e,c + f
\end{pmatrix}\\
&= b + e
\end{align*}
and
\begin{align*}
\varphi(X) + \phi(Y) &= b + e
\end{align*}
Thus,
\[\varphi(X + Y) = \varphi(X) + \varphi(Y) \]
Hence, \( \varphi \) is a homomorphism.
\end{proof}

\qpart

The kernel of \( \varphi \) is,
\[ \begin{pmatrix}
a,0\\
b,c
\end{pmatrix} : a,c \in \Z, b \in 5\Z \]

\qpart

A group that is isomorphic to the quotient group
\( M/Ker(\varphi) \) is \( \Z_{5} \)

\end{question}

%Question 4
\begin{question}


    \[ G = \Z_{101} \times \Z_{50} \]

    \qpart

As \( 101 \text{ and } 50 \) are coprime (\textup{Corollary 2.12, HB Chapter 6, pg.32}),
\[ \Z_{101} \times \Z_{50} \cong \Z_{5050} \]

The order of \( G \) is \( 101 \times 50 = 5050 \).

\qsubpart

The order of the element \( (25,1) \) is;

\begin{align*}
\frac{101}{\hcf(101,25)} &= 101\\
\frac{50}{\hcf(50,1)} &= 50\\
\stext{Thus, the order of (25,1) is } \lcm(101,50) &= 5050
\end{align*}

Hence \( (25,1) \) is a generator of  \( G \).

    \qsubpart

The order of the element \( (1,25) \) is;

\begin{align*}
\frac{101}{\hcf(101,1)} &= 101\\
\frac{50}{\hcf(50,25)} &= 2\\
\stext{Thus, the order of (1,25) is } \lcm(101,2) &= 202
\end{align*}

Hence \( (1,25) \) is not a generator of \( G \).

\qpart

\qsubpart

First group

\[ \Z_{600} \times \Z_{49} \]

The order of this group is \( 600 \times 49 = 29400 \).

Using \textup{Corollary 2.12 and Theorem 2.13, HB Chapter 6, p32},
\begin{align*}
\Z_{600} &\cong \Z_{8} \times \Z_{3} \times \Z_{25}\\
\Z_{49} &\cong \Z_{49}
\end{align*}

Hence,
\[ \Z_{600} \times \Z_{49} \cong \Z_{8} \times \Z_{3} \times \Z_{25} \times \Z_{49} \]

Second group

\[ \Z_{2} \times \Z_{12} \times \Z_{175} \times \Z_{7} \]

The order of this group is \( 2 \times 12 \times 175 \times 7 = 29400 \).

Using \textup{Corollary 2.12 and Theorem 2.13, HB Chapter 6, p32},
\begin{align*}
\Z_{2} &\cong \Z_{2}\\
\Z_{12} &\cong \Z_{4} \times \Z_{3}\\
\Z_{175} &\cong \Z_{25} \times \Z_{7}\\
\Z_{7} &\cong \Z_{7}
\end{align*}    

Therefore a decomposition as a direct product of cyclic groups of prime power order is,

\[ \Z_{2} \times (\Z_{4} \times \Z_{3}) \times (\Z_{25} \times \Z_{7}) \times \Z_{7} \cong \]
\[ \Z_{2} \times \Z_{4} \cong \Z_{3} \]

Thus (By \textup{Theorem 2.13, HB Chapter 6, p32}),
\[ \Z_{2} \times \Z_{12} \times \Z_{175} \times \Z_{7} \cong \Z_{14} \times \Z_{2100} \]

Third group

 \[ \Z_{2} \times \Z_{2} \times \Z_{6} \times \Z_{1225} \]
 
The order of this group is \( 2 \times 2 \times 6 \times 1225 = 29400 \).

Using \textup{Corollary 2.12 and Theorem 2.13, HB Chapter 6, p32},
\begin{align*}
\Z_{2} &\cong \Z_{2}\\
\Z_{6} &\cong \Z_{2} \times \Z_{3}\\
\Z_{1225} &\cong \Z_{25} \times \Z_{49}
\end{align*}

Thus (By \textup{Theorem 2.13, HB Chapter 6, p32}),
\[ \Z_{2} \times \Z_{2} \times \Z_{6} \times \Z_{1225} \cong \Z_{2} \times \Z_{2} \times \Z_{2} \times \Z_{3} \times \Z_{25} \times \Z_{49} \]

\qsubpart

By \textup{Theorem 2.11, HB Chapter 6, p32}, a direct product is cyclic if and only if
the component orders are coprime.

\begin{itemize}
\item First group; \( \Z_{600} \times \Z_{49} \); \( \hcf(600,49) = 1 \) so it is cyclic.
\item Second group; \( \Z_{2} \times \Z_{12} \times \Z_{175} \times \Z_{7} \); \( \hcf(2,12) = 2 \) so it is not cyclic.
\item Third group; \( \Z_{2} \times \Z_{2} \times \Z_{6} \times \Z_{1225} \); \( \hcf(2,6) = 2 \) so it is not cyclic.
\end{itemize}

Therefore only the group \( \Z_{600} \times \Z_{49} \) is cyclic.

\qpart

Since
\[ 1144 = 2^3 \times 11 \times 13 \]
and, Using \textup{Corollary 2.12, HB Chapter 6, p32},
\[ \Z_{11} \times \Z_{13} \cong \Z_{143} \]

Two possible (non-isomorphic) non-cyclic abelian groups of order \( 1144 \) are,

\[ \Z_{1144} \cong \Z_{4} \times \Z_{2} \times \Z_{143} \]
\[ \Z_{1144} \cong \Z_{2} \times \Z_{2} \times \Z_{2} \times \Z_{143} \]

They are non-isomorphic because their 2-power components differ.

\end{question}

%Question 5
\begin{question}

   \[ D_{50} = \langle r,s \mid r^{50} = s^{2} = e, sr = r^{49}s \rangle \]

\qpart

Let \( H = \{ r^{2i}s^{j}:i=0,\dots,24, j=0,1 \} \)
    
Using \textup{Proposition 2.4, HB p25}.

\begin{proof}
$ $\newline

\textbf{a. Non-empty:}

Let \( i=0, j=0 \)
\[ r^{0}s^{0} &= e \in H \]
Thus, H is non-empty.

\vspace{2cm}

\textbf{b. }\( x,y \in H \implies x^{-1}y \in H: \)

Let
\[ x = r^{2a}s^{b} \text{ and } y = r^{2c}s^{d}, \quad a,c\in\{ 0,\ldots,24 \}, \quad b,d\in\{0,1\}  \]
then
\begin{align*}
x^{-1} &= r^{-2a}s^{b}\\[8pt]
\snote{since \( s^{-1} = s \)}\\
x^{-1}y &= r^{-2a}s^{b} \cdot r^{2c}s^{d}\\[8pt]
&= r^{-2a} \cdot s^{b+d} \cdot r^{2c}\\[8pt]
\end{align*}

Let \( j = 0 \),
\begin{align*}
x^{-1}y &= r^{-2a} \cdot r^{2c}\\[8pt]
&= r^{2(c - a)} \in H
\end{align*}

Let \( j = 1 \),
\begin{align*}
x^{-1}y &= r^{-2a}s \cdot r^{2c}s\\[8pt]
&= r^{-2a} \cdot r^{2c} \cdot s^{2}\\[8pt]
\stext{As \( s^{2} = e \), }
&= r^{2(c - a)} \in H
\end{align*}

Hence, \( H \) is a subgroup of \( D_{50} \).

\end{proof}

\end{question}


\section{Part B}



\end{document}