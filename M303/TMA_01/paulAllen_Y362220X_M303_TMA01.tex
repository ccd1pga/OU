\documentclass{tufte-handout}

\usepackage{style}

\begin{document}

\tma{01} % TMA number

\section{Part A}

%Question 1
\begin{question}

Let \[ A = \{1,2,3,4,5\}, \quad B = \{cat,dog,rabbit,hamster\}, \quad C = \mathbb{Z}, \quad D = \{\alpha,\beta,\gamma,\delta\} \]

\qpart

\qsubpart

\[ f: A \rightarrow B, \]
\[ f(1)=cat, f(2)=hamster, f(3)=dog, f(4)=rabbit, f(5)=cat \]
    
 $f$ is onto but not one-to-one.
 
\vspace{2cm}

 \qsubpart

\[ g: A \rightarrow D, \]
\[ g(1)=\alpha, g(2)=\delta, g(3)=\beta, g(4)=\beta, g(5)=\alpha \]

 $g$ is neither one-to-one nor onto
 
\vspace{2cm}

 \qsubpart

 \[ F: A \rightarrow C, \]
\[ F(a) = 100a, \text{ for all } a \in A \]
 
 $F(a)=100a$ is one-to-one but not onto

\vspace{2cm} 

 \qsubpart

 \[ \phi: B \rightarrow D, \]
\[ \phi(cat)=\alpha, \phi(dog)=\gamma, \phi(rabbit)=\beta, \phi(hamster)=\delta \]

 $\phi$ is a bijection.

\vspace{5cm}

\qpart

\qsubpart

 Example of a map \( S\to T \) that is not one-to-one: \[ 2\mapsto 1,\,3\mapsto 1,\,5\mapsto 5,\,7\mapsto 7 \].\\[6pt]

\vspace{2cm}

 \qsubpart

 Example of a bijection \( S\to T \): \[ 2\mapsto 1,\,3\mapsto 3,\,5\mapsto 5,\,7\mapsto 7 \].

\end{question}

%Question 2
\begin{question}

    \[ \text{for } n \geq 1, 7^{2n} - 6^{n} \text{ is divisible by } 43 \]

\begin{proof}
    $ $\newline
Using proof by induction we will first prove the basis of induction \( n = 1 \)

\begin{align*}
    P(1) &= 7^{2(1)} - 6^1\\[8pt]
    &= 49 - 6 \\[8pt]
    &= 43\\[8pt]
    \stext{Hence P(1) is divisible by 43}
\end{align*}

Now we will assume that \( P(k) \) is true for some arbitrary \( k \geq 1 \)
\begin{align*}
    P(k) &= 7^{2k} - 6^k \\[8pt]
    &= 43m\\[8pt]
\stext{ for some integer m}
\end{align*}

Next we will use this as our inductive hypothesis to prove that \( P(k+1) \) is true

\begin{align*}
P(k+1) &= 7^{2(k+1)} - 6^{k+1}\\[8pt]
&= 7^{2k+2} - 6^{k+1}\\[8pt]
&= 7^2 \cdot 7^{2k} - 6 \cdot 6^k\\[8pt]
&= 49 \cdot 7^{2k} - 6 \cdot 6^k\\[8pt]
\stext{From our inductive hypothesis we know that \( 7^{2k} - 6^k = 43m \)}\\[8pt]
&= 49(43m + 6^k) - 6 \cdot 6^k\\[8pt]
&= 49 \cdot 43m + 49 \cdot 6^k - 6 \cdot 6^k\\[8pt]
\stext{Factorising out the \( 6^k \)}
&= 49 \cdot 43m + (49 - 6) \cdot 6^k\\[8pt]
&= 49 \cdot 43m + 43 \cdot 6^k\\[8pt]
\stext{Factoring out the 43}\\[8pt]
&= 43(49m + 6^k)\\[8pt]
\stext{Since \( m \text{ and } k \) are both integers, \( 49m + 6^k \) is also an integer}\\[8pt]
\stext{hence \( P(k+1) \) is divisible by 43}\\[8pt]
\stext{By the principle of mathematical induction we have shown that \( P(n) \) is true for all \( n \geq 1 \)}
\end{align*}

\end{proof}

\end{question}

%Question 3
\begin{question}

\[ \hcf(2025,630) = 2025x +630y \]

\begin{align*}
2025 &= 3 \cdot 630 + 135\\
630 &= 4 \cdot 135 + 90\\
135 &= 1 \cdot 90 + 45\\
90 &= 2 \cdot 45 + 0\\
\stext{Hence, } \hcf(2025,630) = 45\\
\end{align*}

rearranging to make the remainder the subject;
\begin{align*}
45 &= 135 - 1 \cdot 90\\
90 &= 630 - 4 \cdot 135\\
135 &= 2025 - 3 \cdot 630\\
\end{align*}

Substituting back up the chain we get;
\begin{align*}
45 &= 135 - 1(630 - 4 \cdot 135)\\
&= 5 \cdot 135 - 1 \cdot 630\\
&= 5(2025 - 3 \cdot 630) - 1 \cdot 630\\
&= 5 \cdot 2025 - 15 \cdot 630 - 1 \cdot 630\\
&= 5 \cdot 2025 - 16 \cdot 630\\
\stext{Hence one solution is } 
\boxed{x = 5, y = -16}
\end{align*}

\end{question}

%Question 4
\begin{question}

\qpart

If \( a \) and \( b \) are positive integers and  \( p \) divides \( \hcf(a^2,b) \) if, and only if,
\( p \) divides \( \hcf(a,b^2) \).

\begin{proof}
$ $\newline

Let \( p \) be a prime number.

\marginnote{\textup{Definition 4.5, Chapter 1 HB p15}: Highest common factor(HCF), \( \hcf(a,b) \), of two integers \( a \text{ and } b \), not both of which are zero, is the 
natural number \( n \) satisfying \[(a) n \mid a \text{ and } n \mid b; \] \[ \text{if } d \mid a \text{ and } d \mid b \text{ then } d \leq n \]}

Forward direction (\( \Rightarrow \)):
Suppose \( p \mid \hcf(a^2,b) \) then by \textup{Definition 4.5, HB p15},
\[ p \mid a^2 \text{ and } p \mid b \]

Since \( p \) is a prime number, and \( p \mid a^2 \), by Euclid's Lemma for prime factors (\textup{Theorem 1.4, Chapter 2 HB p17}),
\[ p \mid a \]
Hence,
\[ p \mid a \text{ and } p \mid b \]

Which implies, by \textup{Definition 4.5, HB p15},
\[ p \mid \hcf(a,b^2) \because p \mid b \implies p \mid b^2 \]

Backward direction (\( \Leftarrow \)):
conversely suppose \( p \mid \hcf(a,b^2) \) Then
\[ p \mid a \text{ and } p \mid b^2 \]

Again by \textup{Theorem 1.4, HB p17}, 
\[ p \mid b^2 \implies p \mid b \]

Therefore,
\[ p \mid a \text{ and } p \mid b \implies p \mid a^2 \text{ also } \]

Hence \( p \mid(a^2,b) \)

Thus, we have shown in both directions and so;
\[ p \mid \hcf(a^2,b) \Longleftrightarrow p \mid \hcf(a,b^2)  \]

\end{proof}

\vspace{5cm}

\qpart

If \( a \) and \( b \) are positive integers and \( m \) and \( n \) are
defined by 
\[ m = 3a + b \quad \text{ and } \quad n = 5a + 2b  \]
then \( \hcf(m,n) = \hcf(a,b) \)

\begin{proof}
    $ $\newline

Let \( d = \hcf(a,b) \) and \( c = \hcf(m,n) \).

Since \( d \) is the highest common factor of \( a \) and \( b \), by \textup{Definition 4.5, HB p15},
\[ d \mid a \text{ and } d \mid b \]

Then by Properties of division (\textup{Theorem 4.4, Chapter 1 p15}),
\[ d \mid (3a + b) \text{ and } d \mid (5a + 2b) \]
Hence,
\[ d \mid m \text{ and } d \mid n \]
Therefore, by \textup{Definition 4.5},
\[ d \mid c \]  

And since \( c \) is the highest common factor of \( m \) and \( n \), by \textup{Definition 4.5, HB p15},
\[ c \mid m \text{ and } c \mid n \]
Then by \textup{Theorem 4.4},
\[ c \mid (2m - n) \text{ and } c \mid (5m - 3n) \]
Hence,
\[ c \mid a \text{ and } c \mid b \]
Therefore, by \textup{Definition 4.5, HB p15},
\[ c \mid d \]
Thus, \( d \mid c \text{ and } c \mid d \implies c = d \)
\[ \hcf(m,n) = \hcf(a,b) \]

\end{proof}

\vspace{5cm}

\qpart

A number of the form \( 14a + 3 \), where \( a \) is a non-negative integer, must have a prime
divisor of this same form \( 14b + 3 \), where \( b \) is a non-negative integer.

\begin{proof}
    $ $\newline

    Consider if \( n \) is prime, then \( n \) is a prime divisor of itself and since
    \( n = 14a + 3 \), \( n \) is of the form \( 14b + 3 \) where \( b = a \).

    Now consider if \( n \) is composite, then by Fundamental theorem of arithmetic (\textup{Theorem 1.7, Chapter 2 HB p17}), \( n \) can be expressed as;
    \[ n = p_1^{k_1}p_2^{k_2} \ldots p_r^{k_r} \]
    thus having at least one prime factor such that \( p \mid n \)

    Hence we will have to show that;
    \[ n \equiv 3 \pmod{14} \]
        and
    \[ p \mid n, \text {such that } p \equiv 3 \pmod{14} \]

    Consider the counter example where \( n = 185 \), that is;
    \[ n = 185 = 5 \times 37 \]
    \[ 5 \equiv 5 \pmod{14} \]
    and
    \[ 37 \equiv 9 \pmod{14} \]
As neither of these prime factors are of the form \( 14b + 3 \) the statement is incorrect.

\end{proof}

\vspace{5cm}

\qpart

If \( n \) is divisible by \( 15 \) but not divisible by \( 9 \text{ or } 25 \), then \( \tau(n) \)
is divisible by \( 4 \).

\begin{proof}
    $ $\newline

By \textup{Theorem 1.7, HB p17}, we can express \( n \) as a product of its prime factors;
\[ n = 3^{k_1} \cdot 5^{k_2} \cdot p_1^{k_1} \cdot p_2^{k_2} \ldots \]
where \( k_1, k_2 = 1 \) as \( n \) is divisible by \( 15 \) but not by \( 9 \text{ or } 25 \).

Hence,
\[ n = 3^1 \cdot 5^1 \cdot p_1^{k_1} \cdot p_2^{k_2} \ldots \]

Then by \textup{Proposition 2.3, Chapter 2 p17} (Formula for \(\tau(n)\));
\begin{align*}
\tau(n) &= (1 + 1)(1 + 1)(k_1 + 1)(k_2 + 1) \ldots \\[8pt]
&= 2 \cdot 2 \cdot (k_1 + 1)(k_2 + 1) \ldots \\[8pt]
&= 4 \cdot (k_1 + 1)(k_2 + 1) \ldots \\[8pt]
\stext{Since \( (k_1 + 1)(k_2 + 1) \ldots \) is an integer, \( \tau(n) \) is divisible by 4}
\end{align*}

\end{proof}

\end{question}

%Question 5
\begin{question}

\qpart
\qsubpart

\begin{align*}
3x \equiv 4 \pmod{5}\\[8pt]
\stext{\( \hcf(3,5) = 1 \), hence \( 3 \) has a multiplicative inverse modulo \( 5 \)}\\
\stext{Hence, this will be the unique solution to the linear congruence}
3v \equiv 1 \pmod{5}\\[8pt]
\stext{By trying values for \( v \) we find that;}
3 \cdot 2 \equiv 1 \pmod{5}\\[8pt]
\stext{Thus, \( v = 2 \) is the multiplicative inverse of \( 3 \) modulo \( 5 \)}\\
\stext{Multiplying both sides of the original congruence by \( 2 \) gives;}
6x \equiv 8 \pmod{5}\\[8pt]
\stext{Reducing \( 6 \) and \( 8 \) modulo \( 5 \) gives;}
x \equiv 3 \pmod{5}\\[8pt]
\stext{Thus, the solution is;}
\boxed{x \equiv 3 \pmod{5}}
\end{align*}

\vspace{3cm}

\qsubpart

\begin{align*}
3x -1 \equiv 2(4 + x) \pmod{7}\\[8pt]
\stext{distributing the \( 2 \) on the RHS;}
3x - 1 \equiv 8 + 2x \pmod{7}\\[8pt]
\stext{subtracting \( 2x \) from both sides;}
x - 1 \equiv 8 \pmod{7}\\[8pt]
\stext{adding \( 1 \) to both sides;}
x \equiv 9 \pmod{7}\\[8pt]
\stext{reducing \( 9 \) modulo \( 7 \) gives;}
x \equiv 2 \pmod{7}\\[8pt]
\stext{Thus, the solution is;}
\boxed{x \equiv 2 \pmod{7}}
\end{align*}

\vspace{3cm}

\qsubpart

\begin{align*}
2(7 - x) \equiv 8 - x \pmod{17}\\[8pt]
\stext{distributing the \( 2 \) on the LHS;}
14 - 2x \equiv 8 - x \pmod{17}\\[8pt]
\stext{adding \( 2x \) to both sides;}
14 \equiv 8 + x \pmod{17}\\[8pt]
\stext{subtracting \( 8 \) from both sides;}
6 \equiv x \pmod{17}\\[8pt]
\stext{Thus, the solution is;}
\boxed{x \equiv 6 \pmod{17}}
\end{align*}

\vspace{5cm}

\qpart

\[ x \equiv 3 \pmod{5} \]
\[ x \equiv 2 \pmod{7}  \]
\[ x \equiv 6 \pmod{17} \]
\[\hcf(5,7)=\hcf(5,17)=\hcf(7,17)=1\]

Since $5$, $7$, and $17$ are coprime, by the Chinese Remainder Theorem (CRT, \textup{Theorem 4.2, Chapter 3 p20}) there is a unique 
solution modulo $5\times7\times17=595$. Checking shows $x=23$ satisfies all three congruences, 
so the least positive solution is;

$\boxed{x=23}$.

\end{question}

%Question 6
\begin{question}

    \qpart

    \[ 25^{60} \pmod{59} \]

Using Fermat's little theorem (FLT, \textup{Theorem 1.1, Chapter 4 p21});

\begin{align*}
\stext{\( a^{p-1} \equiv 1 \pmod{p} \) where \( p \) is a prime and \( a \) is not divisible by \( p \)}\\[8pt]
25^{58} &\equiv 1 \pmod{59}\\[8pt]
\stext{Substitute this into the original expression, and using index laws;}\\[8pt]
&\equiv 25^{58} \cdot 25^2 \pmod{59}\\[8pt]
\stext{Reducing \( 25^{58} \) modulo \( 59 \) gives;}\\[8pt]
&\equiv 1 \cdot 25^2 \pmod{59}\\[8pt]
&\equiv 625 \pmod{59}\\[8pt]
\stext{Reducing \( 625 \) modulo \( 59 \) gives;}
&\equiv 35 \pmod{59}\\[8pt]
\stext{Thus, the solution is;}
\boxed{35}
\end{align*}

\vspace{5cm}

\qpart

\[ a^{25} \equiv \pmod{195} \]

\begin{proof}
$ $\newline

Since \( 195 = 3 \times 5 \times 13 \) and these are coprime, we can use CRT and show;
\[ a^{25} \equiv a \pmod{p} \]
For each of the primes divisors \( p \in \{3,5,13\} \)

Using FLT we need to check that \( 25 \equiv 1 \pmod{p-1} \) for each prime:

\begin{align*}
    p=3: & 25 \equiv 1 \pmod{2}\\[8pt]
    p=5: & 25 \equiv 1 \pmod{4}\\[8pt]
    p=13: & 25 \equiv 1 \pmod{12}\\[8pt]
\end{align*}

Hence, for each prime \( p \) dividing \( 195 \),
\[ a^{25} \equiv a^1 \equiv a \pmod{p} \]

By the CRT, there is a unique solution modulo \( 195 \) and so;
\[ a^{25} \equiv a \pmod{195} \]    
for all integers \( a \).

\end{proof}

\end{question}

%Question 7
\begin{question}

\[ P(x) = x^3 + 23x^2 + 10x + 6 \]

\qpart

The degree of \( P(x) \) is \( 3 \).

\vspace{2cm}

\qpart

\qsubpart

\[ P(x) \equiv \pmod{3} \]

\begin{align*}
P(x) &\equiv x^3 + 23x^2 + 10x + 6 \pmod{3}\\[8pt]
&\equiv x^3 + 2x^2 + x \pmod{3}\\[8pt]
\stext{Factoring out an \( x \) gives;}\\[8pt]
&\equiv x(x^2 + 2x + 1) \pmod{3}\\[8pt]
\stext{Factor out \(x\) and complete the square;}\\[8pt]
&\equiv x\bigl(x^2 + 2x + 1\bigr) \pmod{3}\\[8pt]
&\equiv x(x+1)^2 \pmod{3}\\[8pt]
\stext{Thus the roots satisfy \(x \equiv 0 \) or \( x \equiv -1 \equiv 2 \pmod{3}\).}\\[8pt]
\boxed{x \equiv 0, 2 \pmod{3}}
\end{align*}

\vspace{2cm}

\qsubpart

\[ P(x) \equiv 0 \pmod{7} \]

\begin{align*}
P(x) &\equiv x^3 + 23x^2 + 10x + 6 \pmod{7}\\[8pt]
&\equiv x^3 + 2x^2 + 3x + 6 \pmod{7}\\[8pt]
\stext{Trying values for \( x \) from \( 0 \) to \( 6 \) gives;}\\[8pt]
x = 2 &\equiv 0 \pmod{7}\\[8pt]
x = 5 &\equiv 0 \pmod{7}\\[8pt]
\stext{Thus, the solutions are;}\\[8pt]
\boxed{x \equiv 2, 5 \pmod{7}}
\end{align*}

\vspace{5cm}

\qpart

\[ P(x) \equiv 0 \pmod{63} \]

Using the CRT we can split this into two congruences;
\[ P(x) \equiv 0 \pmod{7} \]
\[ P(x) \equiv 0 \pmod{9} \]
\[\hcf(7,9)=1\]

\begin{align*}
P(x) &\equiv x^3 + 23x^2 + 10x + 6 \pmod{9}\\[8pt]
&\equiv x^3 + 5x^2 + x + 6 \pmod{9}\\[8pt]
\stext{Trying values for \( x \) from \( 0 \) to \( 8 \) gives;}\\[8pt]
x = 2 &\equiv 0 \pmod{9}\\[8pt]
x = 3 &\equiv 0 \pmod{9}\\[8pt]
x = 5 &\equiv 0 \pmod{9}\\[8pt]
x = 8 &\equiv 0 \pmod{9}\\[8pt]
\stext{Thus, the solutions are;}\\[8pt]
\boxed{x \equiv 2, 3, 5, 8 \pmod{9}}
\end{align*}

Using the solutions from part (b) we have;
\[ x \equiv 2, 5 \pmod{7} \]
\[ x \equiv 2, 3, 5, 8 \pmod{9} \]
\[\hcf(7,9) = 1\]

Since $7$ and $9$ are coprime, by the CRT there is a unique solution modulo $7\times9=63$.

\begin{align*}
\stext{first pair}
x &\equiv 2 \pmod{7}\\[8pt]
x &\equiv 2 \pmod{9}\\[8pt]
\implies x &\equiv 2 \pmod{63}
\end{align*}

Now using the multiplicative inverse \( v \) of \( 7 \pmod{9} \);
\[ v = 4, \text{ since } 7 \times 4 \equiv 1 \pmod{9} \]

\begin{align*}
\stext{second pair}
x &\equiv 2 \pmod{7}\\[8pt]
x &\equiv 3 \pmod{9}\\[8pt]
x &= 7k +2\\[8pt]
7k + 2 &\equiv 3 \pmod{9}\\[8pt]
\stext{Substituting \( v \) and multiplying both sides by \( 4 \) gives;}\\[8pt]
4(7k + 2) &\equiv 4 \cdot 3 \pmod{9}\\[8pt]
k + 8 &\equiv 12 \pmod{9}\\[8pt]
k &\equiv 4 \pmod{9}\\[8pt]
\stext{Substituting back gives;}\\[8pt]
x &= 7(9m + 4) + 2\\[8pt]
&= 63m + 30\\[8pt]
\implies x &\equiv 30 \pmod{63}\\[8pt]
\end{align*}

Checking the rest of the pairs, shows \( x=2, 5, 16, 23, 30, 37, 47, 54 \) satisfy both congruences, 
so the least positive solutions are;

$\boxed{2, 5, 16, 23, 30, 37, 47, 54}$.

\end{question}

%Question 8
\begin{question}    

\qpart

I have watched the online session entitled "M303: Things you need to know".

\vspace{2cm}

    \qpart

    \includegraphics[scale=0.5]{question_8_b.jpg}

https://learn2.open.ac.uk/mod/forumng/discuss.php?d=5080556

\end{question}

\section{Part B}

%Question 10
\begin{question}
    Question 10 from the TMA booklet

    \qpart

    For \( k \geq 1 \) let \( P(n) \) be the statement;

"Let \( k \) be the number of blue eared aliens, all the aliens will leave at
\( 6am \) on the \( k^{th} \) day."

\vspace{5cm}

    \qpart

    \textbf{Base case:}
    \[ P(1): k = 1 \]

    If a blue eared alien does not see any other blue eared aliens, then they will know they are the only 
    blued eared alien on the planetoid, and will leave at \( 6am \) on the next rotation, i.e the \( k^{th} \) day.

So \( P(1) \) holds

    \textbf{Inductive step:}
    Assume that the statement is true for some arbitrary \( m \geq 1 \), i.e if there are \( m \) blue eared aliens, 
    they will all leave at \( 6am \) on the \( m^{th} \) day.

    Now consider if there are \( m + 1 \) blue eared aliens. Each blue eared alien will see \( m \) 
    other blue eared aliens. When on the \( m^{th} \) day no aliens leave, they will realise that they must also
    have blue ears, so all \( m + 1 = k \) aliens will leave at \( 6am \) on the \( k^{th} \) day.

    Thus, by the principle of mathematical induction, the statement is true for all \( k \geq 1 \).

\end{question}

%Question 12
\begin{question}
    Question 12 from the TMA booklet

Let \( p \text{ and } q \) be distinct odd primes.

\qpart

An integer \( x \) satisfied \( x^2 \equiv 1 \pmod{pq} \) if, and only if, \( x \) satisfies both
\[ x^2 \equiv 1 \pmod{p} \text{ and } x^2 \equiv 1 \pmod{q} \]

\begin{align*}
\stext{Forward direction (\( \Rightarrow \)):}\\[8pt]
x^2 &\equiv 1 \pmod{pq}\\[8pt]
x^2 - 1 &\equiv 0 \pmod{pq}\\[8pt]
(x - 1)(x + 1) &\equiv 0 \pmod{pq}\\[8pt]
\end{align*}

Hence, \( pq \mid (x - 1)(x + 1) \)
Since \( p \text{ and } q \) are distinct primes;

\begin{align*}
p &\mid (x - 1)(x + 1) \text{ and } q \mid (x - 1)(x + 1)\\[8pt]
\stext{Thus,}\\[8pt]
x^2 &\equiv 1 \pmod{p} \text{ and } x^2 \equiv 1 \pmod{q}\\[8pt]
\end{align*}

\begin{align*}
\stext{Backward direction (\( \Leftarrow \)):}\\[8pt]
x^2 &\equiv 1 \pmod{p} \text{ and } x^2 \equiv 1 \pmod{q}\\[8pt]
x^2 - 1 &\equiv 0 \pmod{p} \text{ and } x^2 - 1 \equiv 0 \pmod{q}\\[8pt]    
(x - 1)(x + 1) &\equiv 0 \pmod{p} \text{ and } (x - 1)(x + 1) \equiv 0 \pmod{q}\\[8pt]
\stext{Hence,}\\[8pt]
p &\mid (x - 1)(x + 1) \text{ and } q \mid (x - 1)(x + 1)\\[8pt]
\stext{Since \( p \text{ and } q \) are distinct primes;}\\[8pt]
pq &\mid (x - 1)(x + 1)\\[8pt]
(x - 1)(x + 1) &\equiv 0 \pmod{pq}\\[8pt]
x^2 - 1 &\equiv 0 \pmod{pq}\\[8pt]
x^2 &\equiv 1 \pmod{pq}\\[8pt]
\end{align*}

Therefore,

\boxed{x^2 \equiv 1 \pmod{pq} \Longleftrightarrow x^2 \equiv 1 \pmod{p} \text{ and } x^2 \equiv 1 \pmod{q}}

\vspace{5cm}

\qpart

\[ x^2 \equiv 1 \pmod{p} \]

As \( p \) is odd, \( p - 1 \) is even, hence divisible by \( 2 \).

By \textup{Proposition 4.8, Chapter 4 p22}, if \( p \) is prime and \( d \) is a textupof \( p - 1 \) then the
congruence \( x^d -1 \equiv 0 \pmod{p} \) has exactly \( d \) solutions.

Thus, \( x^2 \equiv 1 \pmod{p} \) has exactly \( 2 \) solutions.
These solutions are \( x \equiv 1 \pmod{p} \) and \( x \equiv -1 \equiv p - 1 \pmod{p} \).

\vspace{5cm}

\qpart

\[ x^2 \equiv 1 \pmod{pq} \]

\[ x^2 \equiv 1 \pmod{p} \]
\[ x^2 \equiv 1 \pmod{q} \]

\begin{align*}
x^2 &\equiv 1 \pmod{p}\\[8pt]
x^2 - 1 &\equiv 0 \pmod{p}\\[8pt]
(x - 1)(x + 1) &\equiv 0 \pmod{p}\\[8pt]
\stext{Hence, \( p \mid (x - 1) \text{ or } p \mid (x + 1) \)}\\[8pt]
\stext{Thus, the solutions are;}\\[8pt]
x &\equiv 1 \pmod{p} \text{ or } x \equiv -1 \equiv p - 1 \pmod{p}\\[8pt]
\end{align*}

The same argument can be used for \( q \). Hence, 
\[ x^2 \equiv 1 \pmod{pq} \text{ has four solutions}. \]

\vspace{5cm}

\qpart

\[ x^2 \equiv 1 \pmod{95} \]
Using the CRT we can split this into two congruences;
\[ x^2 \equiv 1 \pmod{5} \]
\[ x^2 \equiv 1 \pmod{19} \]
\[\hcf(5,19)=1\]

Since $5$ and $19$ are coprime, by the CRT there is a unique
solution modulo $5\times19=95$.

The solutions to \( x^2 \equiv 1 \pmod{5} \) are;
\[ x \equiv 1 \pmod{5} \text{ or } x \equiv 4 \pmod{5} \]

The solutions to \( x^2 \equiv 1 \pmod{19} \) are;
\[ x \equiv 1 \pmod{19} \text{ or } x \equiv 18 \pmod{19} \]

Checking the four pairs of solutions shows $x=1, 4, 16, 19$ satisfy both congruences, so the least positive solutions are;

$\boxed{1, 4, 16, 19}$.

\vspace{5cm}

\qpart

This does not contradict Lagrange's theorem (\textup{Theorem 4.3, Chapter 4 p22}) as this theorem only applies to
polynomials modulo a prime number, and \( 95 \) is not prime.

\end{question}

\end{document}