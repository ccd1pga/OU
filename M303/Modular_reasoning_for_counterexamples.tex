\documentclass{tufte-handout}

\usepackage{style}

\begin{document}

How to use this in practice
	•	Stick it near the top of your number theory notes under something like 
    “Modular reasoning for counterexamples”.
	•	When you get a new problem, jot down the modulus and target residue, make 
    the residue multiplication table for that modulus, and check whether the 
    condition is even plausible before attempting a proof.

\begin{section}{Reasoning with Modular Arithmetic for Proof or Counterexample}

When faced with a statement of the form:
\[
\text{For all integers } n \text{ such that } n \equiv r \pmod{k}, 
\text{ there exists a prime divisor } p \text{ such that } p \equiv r \pmod{k},
\]
it is often not obvious whether the statement is \emph{true} or \emph{false}.  
Rather than testing individual numbers, it is usually more efficient to reason using 
\emph{residue classes modulo \(k\)}.

\subsection*{Step 1: Translate to modular form}

Let \( n = \prod_{i=1}^t p_i^{\alpha_i} \) be the prime factorisation of \( n \).
Then
\[
n \equiv \prod_{i=1}^t (p_i \bmod k)^{\alpha_i} \pmod{k}.
\]
Hence, the residue of \(n\) modulo \(k\) depends only on the residues of its prime factors.

If the statement claims that one of the factors must have a specific residue,
you can test whether that residue is \emph{necessary} for the product to have a certain value.

\subsection*{Step 2: Identify possible residues of primes modulo \(k\)}

For a given modulus \(k\), the primes \(p\) (other than those dividing \(k\)) 
can only take residues coprime to \(k\).  
Define the reduced residue system modulo \(k\) as:
\[
R = \{ r \in \{1, 2, \ldots, k-1\} : \gcd(r, k) = 1 \}.
\]

\marginnote{Example: For \(k = 14\), \(R = \{1, 3, 5, 9, 11, 13\}\).}

\subsection*{Step 3: Construct the multiplication table of residues in \(R\)}

Compute all possible products \(r_i r_j \bmod{k}\) for \(r_i, r_j \in R\).
This shows which residues can arise as products of primes in \(R\).

If a particular residue (such as \(r = 3\)) can be expressed as a product of 
other residues without using \(r\) itself, then it is \emph{not necessary}
for any prime factor to be congruent to \(r \pmod{k}\).

\marginnote{In such a case, a counterexample exists.}

\subsection*{Step 4: Example with \(k = 14\)}

We compute the products of elements of \(R = \{1, 3, 5, 9, 11, 13\}\) modulo 14.

\[
\begin{array}{c|cccccc}
\times & 1 & 3 & 5 & 9 & 11 & 13 \\\hline
1 & 1 & 3 & 5 & 9 & 11 & 13 \\
3 & 3 & 9 & 1 & 13 & 5 & 11 \\
5 & 5 & 1 & 11 & 3 & 13 & 9 \\
9 & 9 & 13 & 3 & 11 & 1 & 5 \\
11& 11& 5 & 13 & 1 & 9 & 3 \\
13& 13& 11& 9 & 5 & 3 & 1 \\
\end{array}
\]

From this table, we see that
\[
5 \times 9 \equiv 3 \pmod{14}, \quad 11 \times 13 \equiv 3 \pmod{14}.
\]
Therefore, a product can be congruent to \(3 \pmod{14}\) even if 
none of its prime factors are congruent to \(3 \pmod{14}\).

\marginnote{Hence, the original claim that every \(n \equiv 3 \pmod{14}\) 
must have a prime divisor \(p \equiv 3 \pmod{14}\) is false.}

\subsection*{Step 5: General conclusion}

To test similar statements:

\begin{enumerate}
  \item Express the statement in modular form.
  \item Determine the set of possible residues of primes modulo \(k\).
  \item Examine whether the target residue can be written as a product of other residues.
  \item If yes, construct a counterexample.
  \item If no, the statement may hold (and you can attempt a proof).
\end{enumerate}

This method avoids brute-force computation by working directly with 
modular structures, providing a clean, logical way to decide 
whether a “prove or counterexample” question is worth proving or disproving.

\end{section}


\end{document}