\documentclass{tufte-handout}

\usepackage{style}

\begin{document}
\bibliography{biblio.bib} 
\bibliographystyle{plainnat}
\tma{01}


%question one
\begin{question}

\end{question}


%question two
\begin{question}

\end{question}

%question three
\begin{question}

\qpart
Passengers could be an entity for Table 1. This is because they are pysical objects and as such are tangible entities.

Ship could be an entity for Table 1. The name of a ship is is also tangible, hence a entity.

\qpart

passengers
\begin{tabular}[]{|c|c|c|}
\hline
    customer\textunderscore ID &  Forename & Surname \\
\hline
\end{tabular}

Each customer is given their own ID number to uniquely identify them. The use of surename or forename is inappropriate as
They are not necessarily unique to each customer.

Ship
\begin{tabular}[]{|c|c|c|}
\hline
        ship\textunderscore ID &  Name & Month \\
\hline
\end{tabular}

The table Ship, has a 'ship\textunderscore ID' which is unique to each ship and when it sails. The name of the ship is not unique 
as the same ship could be ttraveling a number of times a year. The month is not a good attribute as it does not uniquely identify a ship.

\qpart

\begin{tabular}[]{|c|c|}
\hline
    custormer\textunderscore ID & ship\textunderscore ID \\
\hline
\end{tabular}

\qpart

Flat databases, such as Table 1 are not as efficient as relational databases. One reason for this is that they only at the 
moment the table can only accomodate for three voyages per year. This is not efficient as it does not allow for
the possibility of more voyages. unsless you expand the table and then there wouod potentially be lots of empty rows, and hence lots 
of wasted data storage.

Another reason is that the table is not normalised. This means that there are redundancies in the data. For example,
the ship name is repeated for each voyage. This means that if the name of the ship changes, it would have to be changed in multiple places.

\end{question}

%question four
\begin{question}
\qpart




\end{question}

\end{document}