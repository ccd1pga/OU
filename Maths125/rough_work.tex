\documentclass{article}

\usepackage{amsmath}
\usepackage{amssymb}

\begin{document}

\Begin{question}
f=15N right up $\ang{20}$ to horizontal
g=11N right down $\ang{50}$ to horizontal

Answer
H= 21.4N $\ang{81}$ to vertical

\end{question}

\begin{question}
f=80N left up $\ang{30}$ to vertical
g directeted vertally down
h= right down $\ang{45}$ to horizontal

Answer
G= 56.6N
H= 109.3N
\end{question}

\begin{question}

two wires, left wire $\ang{30}$ to vertical, tension 73N
right wire $\ang{45}$ to vertical

Answer 
Mass of sign 10Kg

\end{question}

\begin{question}

mass 10kg on a rough plane $\ang{25}$ to horizontal

Answer

mag of friction = 41N

\end{question}

\begin{question}

    f= linear transformation
\(\begin{pmatrix}
    -4,-2\\
    4,3
\end{pmatrix}\)

answer
f scales by factor -4

\end{question}

\begin{question}

rotation of \( frac{-2\pi}{3} \)
\(\begin{pmatrix}
    a, b\\
    c, -d
\end{pmatrix}\)

Answer
\(\begin{pmatrix}
    \frac{{-1}{2}, \frac{\sqrt{3}}{2}}\\
    \frac{-\sqrt{3}}{2}, \frac{-1}{2}
\end{pmatrix}\)
\end{question}


\begin{question}

    flattening transformation
\( f(x,y) = (3xx-3y,y-x) \)

Answer
\(y=\frac{-1}{3}x\)

\end{question}

\begin{question}

\(
\begin{pmatrix}
-2,-3\\
1,2
\end{pmatrix}
\)

find the point such that
\(f(x,y) = (-4,-4)\)

Answer
\( (20,-12) \)

\end{question}

\begin{question}

\(
\begin{pmatrix}
    3,-4\\
    -2,-2
\end{pmatrix}
\)
the equation of the image of the unit circle

Answer

\( \frac{2}{49}x^2 -\frac{1}{49}xy+ \frac{25}{196}y^2=1 \)

\end{question}

\begin{question}
\( \frac{5u^2 +3u +4}{(u+1)(u^2+u+2)} \)

Answer
\( \frac{3}{u+1} + \frac{2u22}{u^2+u+2} \)
\end{question}

\begin{question}
\( 8x^3-12x^2+8x-1 \) divided by \( 4x-2 \)

Answer
\( 2x^2+2x+1 \) remainder \(1\)
\end{question}

\begin{question}

\( \int \frac{14-2x}{(x-4)(x-1)} \)

Answer
\( 2\ln(x-4) -4\ln(4-1)+c \)

\end{question}

\begin{question}

\( \int \sin(2u)cos(7u ) \)

Answer
\( -\frac{1}{10}\cos(-5u) - \frac{1}{18}\cos(9u) + c \)

\end{question}

\begin{question}

\( \int \frac{6}{\sqrt{16-25x^2}} \)

Answer
\( \frac{6}{5}\arcsin(\frac{5}{4}x) \)

\end{question}

\begin{question}

\( \int \cosh^4(x)sinh^5(x) \)

Answer

    \( \cos^2(2x) -2\cosh(2x) +1 \)

\end{question}

\end{document}