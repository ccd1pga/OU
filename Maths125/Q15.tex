\documentclass{tufte-handout}

\usepackage{style}
\usepackage{polynom}

\newenvironment{amatrix}[1]{%
  \left(\begin{array}{@{}*{#1}{c}|c@{}}
}{%
  \end{array}\right)
}

\begin{document}
\bibliography{biblio.bib} 
\tma{02}

%question one

\begin{question}

\qpart
\qsubpart

\[ g(x,y) = (2x,7y) \]

\begin{align*}
    \begin{pmatrix}
        a & b \\
        c & d
    \end{pmatrix}
    \cdot
    \begin{pmatrix}
        x \\
        y
    \end{pmatrix}
    &=
    \begin{pmatrix}
        2x\\
        7y
    \end{pmatrix}\\[8pt]
    \begin{pmatrix}
        ax + by \\
        cx + dy
    \end{pmatrix}
    &=
    \begin{pmatrix}
        2x + 0y\\
        0x + 7y
    \end{pmatrix}\\[8pt]
    &=
    \begin{pmatrix}
        2 & 0 \\
        0 & 7
    \end{pmatrix}\\
\end{align*}

This is a diagonal scaling, scale by \( 2 \) in \( x‐direction \), \( 7 \) in \( y‐direction \).

\vspace{3cm}
\qsubpart

\[ h(x,y) = (x,4x+y) \]

\begin{align*}
    \begin{pmatrix}
        a & b \\
        c & d
    \end{pmatrix}
    \cdot
    \begin{pmatrix}
        x \\
        y
    \end{pmatrix}
    &=
    \begin{pmatrix}
        x + y\\
        4x + y
    \end{pmatrix}\\[8pt]
    \begin{pmatrix}
        ax + by \\
        cx + dy
    \end{pmatrix}
    &=
    \begin{pmatrix}
        x + 0y\\
        4x + y
    \end{pmatrix}\\[8pt]
    &=
    \begin{pmatrix}
        1 & 0\\
        4 & 1
    \end{pmatrix}\\
\end{align*}

This is a shear transformation.

\vspace{3cm}
\qsubpart

\[ k(x,y) = (y,x) \]

\begin{align*}
    \begin{pmatrix}
        a & b \\
        c & d
    \end{pmatrix}
    \cdot
    \begin{pmatrix}
        x \\
        y
    \end{pmatrix}
    &=
    \begin{pmatrix}
        y\\
        x
    \end{pmatrix}\\[8pt]
    \begin{pmatrix}
        ax + by \\
        cx + dy
    \end{pmatrix}
    &=
    \begin{pmatrix}
        0x + y\\
        x + 0y
    \end{pmatrix}\\[8pt]
    &=
    \begin{pmatrix}
        0 & 1\\
        1 & 0
    \end{pmatrix}\\
\end{align*}

This swaps \( x \) and \( y \).

\vspace{3cm}
\qpart

\[ g = \begin{pmatrix}
    2 & 0 \\
    0 & 7
    \end{pmatrix}
    \qquad
    h = \begin{pmatrix}
        1 & 0 \\
        4 & 1
    \end{pmatrix}
    \qquad
    k = \begin{pmatrix}
        0 & 1 \\
        1 & 0
    \end{pmatrix} 
\]

\[ f = k \circ h \circ g \]

\begin{align*}
    \stext{First we find \( h \circ g \) }\\
    &= h \circ g \\
    &= \begin{pmatrix}
        1 & 0 \\
        4 & 1
    \end{pmatrix}
    \cdot
    \begin{pmatrix}
        2 & 0 \\
        0 & 7
    \end{pmatrix}\\[8pt]
    &= \begin{pmatrix}
        1 \cdot 2 + 0 \cdot 0 & 1 \cdot 0 + 0 \cdot 7 \\
        4 \cdot 2 + 1 \cdot 0 & 4 \cdot 0 + 1 \cdot 7
    \end{pmatrix}\\
    &= \begin{pmatrix}
        2 & 0 \\
        8 & 7
    \end{pmatrix}\\
\stext{Now we find \( k \circ h \circ g \)}\\
    \mathbf{A}  &= k \circ \begin{pmatrix}
        2 & 0 \\
        8 & 7
    \end{pmatrix}\\[8pt]
    &= \begin{pmatrix}
        0 & 1 \\
        1 & 0
    \end{pmatrix}
    \cdot
    \begin{pmatrix}
        2 & 0 \\
        8 & 7
    \end{pmatrix}\\[8pt]
    &= \begin{pmatrix}
        0 \cdot 2 + 1 \cdot 8 & 0 \cdot 0 + 1 \cdot 7 \\
        1 \cdot 2 + 0 \cdot 8 & 1 \cdot 0 + 0 \cdot 7
    \end{pmatrix}\\[8pt]
    &= \begin{pmatrix}
        8 & 7 \\
        2 & 0
    \end{pmatrix}\\
    \snote{As required}
\end{align*}

\vspace{3cm}
\qpart

\marginnote{To find the determiate of a matrix, we can use the formula:\[ Det A = ad - bc \]}
\marginnote{To find th inverse of a matrix we use \[ \mathbf{A^{-1}} = \frac{1}{Det A}\begin{pmatrix}
        d & -b \\
        -c & a
    \end{pmatrix} \]}


First we have to find the determinant of \( f \).

\begin{align*}
    Det A &= Det \begin{pmatrix}
        8 & 7 \\
        2 & 0
    \end{pmatrix}\\[8pt]
    &= 8 \cdot 0 - 7 \cdot 2\\
    &= -14
\end{align*}

As \( Det A \neq 0 \) \( f \) is inversable 

\begin{align*}
    \stext{Hence the matrix that represents \( f^{-1} \) is}
    f^{-1} &= \frac{1}{Det A} \cdot \begin{pmatrix}
        0 & -7 \\
        -2 & 8
    \end{pmatrix}\\[8pt]
    &= \frac{-1}{14} \cdot \begin{pmatrix}
        0 & -7 \\
        -2 & 8
    \end{pmatrix}\\[8pt]
    &= \frac{1}{14} \cdot \begin{pmatrix}
        0 & 7 \\
        2 & -8
    \end{pmatrix}\\[8pt]
    &= \begin{pmatrix}
        0 & \frac{1}{2} \\
        \frac{1}{7} & -\frac{4}{7}
    \end{pmatrix}\\
\end{align*}

\vspace{3cm}
\qpart

First to find the coordinates of the point in the domain of \( f \) 
that is mapped to a general point \( (x,y) \) in the codoimain of \( f \).

\begin{align*}
    \stext{Each point \( (x,y) \) is the image under \( f \) of the point
    \( f^{-1}(x,y) \) }
    \mathbf{A^{-1}}\begin{pmatrix}
        x \\
        y
    \end{pmatrix}
    &= \begin{pmatrix}
        0 & \frac{1}{2} \\
        \frac{1}{7} & -\frac{4}{7}      
    \end{pmatrix}
    \cdot
    \begin{pmatrix}
        x \\
        y
    \end{pmatrix}\\[8pt]
    &= \begin{pmatrix}
        0 \cdot x + \frac{1}{2} \cdot y \\
        \frac{1}{7} \cdot x - \frac{4}{7} \cdot y
    \end{pmatrix}\\[8pt]
\stext{Hence  \( f \) maps the point \( (\frac{y}{2}, \frac{x - 4y}{7}) \)
to the point \( (x,y) \) }
\end{align*}

\marginnote{The general equation of the unit circle is \[ x^{2} + y^{2} = 1 \]}

\begin{align*}
\stext{Substitute these values into the unit circle to find the equation of the image \( f\rb{C} \)}\\[8pt]
    \rb{\frac{y}{2}^{2} + \frac{x - 4y}{7}^{2}} &= 1\\[8pt]
\stext{Mulitplying out th brackets}\\[8pt]
    \frac{y^{2}}{4} + \frac{x^{2} - 8xy + 16y^{2}}{49} &= 1\\[8pt]
\stext{Multiplying through by \( 196 \)}\\[8pt]
    49y^{2} + 4x^{2} - 32xy + 64y^{2} &= 196\\[8pt]
\stext{Combining like terms, leaves us with the equation of \( f\rb{C} \)}
    \frac{1}{196}\rb{4x^{2} - 32xy + 113y^{2}} &= 1\\[8pt]
\stext{Putting this into the form \( ax^{2} + bxy + cy^{2} = d \)}
    4x^{2} - 32xy + 113y^{2} &= 196\\[8pt]
\stext{This is the equation of the image \( f\rb{C} \)}
\stext{where \( a = 4, b = -32, c = 113, d = 196 \)}
\end{align*}

\vspace{3cm}
\qpart

The are of the unit circle is \( \pi \) the area of \( f\rb{C} \) is given by the fact
that linear transformations scale the area by the absolute value of the determinant of 
the transformation matrix.

\begin{align*}
    \stext{The area of the image \( f\rb{C} \) is}
    Area\rb{f\rb{C}} &= \vert Det f \vert \cdot Area\rb{C}\\[8pt]
    &= 14 \cdot \pi\\
    &= 14\pi   
\end{align*}

\end{question}

\clearpage

%question two
\begin{question}

    \qpart

    \polylongdiv{5x^{3}-11x^{2}-99x-72}{x^{2}-3x-18}   

\begin{align*}
\stext{Hence, the quotient is}
    Q(x) &= 5x + 4\\[8pt]
\stext{The remainder is}
    R(x) &= \frac{3x}{x^{2}-3x-18}\\[8pt]
\stext{Writing as partilar fractions}
    \frac{3x}{x^{2}-3x-18} &= \frac{A}{x-6} + \frac{B}{x+3}\\[8pt]
\stext{Multiplying through by \( x^{2}-3x-18 \)}
    3x &= A\rb{x+3} + B\rb{x-6}\\[8pt]
    &= Ax + 3A + Bx - 6B\\[8pt]
\stext{Combining like terms}
    &= \rb{A + B}x + \rb{3A - 6B}\\[8pt]
\stext{Using augmented matrix to solve for \( A \) and \( B \)}
    \begin{amatrix}{2}
        3 & -6 & 0\\
        1 & 1 & 3
    \end{amatrix}\\
    \snote{subtract 3 \( \times \) R2 from R1}\\
    \begin{amatrix}{2}
        0 & -9 & -9\\
        1 & 1 & 3
    \end{amatrix}\\
    \snote{Divide R1 by -9}\\
    \begin{amatrix}{2}
        0 & 1 & 1\\
        1 & 1 & 3
    \end{amatrix}\\
    \snote{Subtract R1 from R2}\\
    \begin{amatrix}{2}
        0 & 1 & 1\\
        1 & 0 & 2
    \end{amatrix}\\
\stext{hence we can write}
    \frac{3x}{x^{2}-3x-18} &= \frac{2}{x-6} + \frac{1}{x+3}\\[8pt]
\stext{Giving}
    \frac{x^{3}-11x^{2}-99x-72}{x^{2}-3x-18} &= 5x + 4 + \frac{2}{x-6} + \frac{1}{x+3}\\[8pt]
\end{align*}

Using this we can solve 
\[ \int\frac{x^{3}-11x^{2}-99x-72}{x^{2}-3x-18}\dif{x} \]

\begin{align*}
    \int\frac{x^{3}-11x^{2}-99x-72}{x^{2}-3x-18}\dif{x} &= \int5x + 4 + \frac{3x}{x^{2}-3x-18}\\[8pt]
    &= \int5x + 4 \dif{x} + \int\frac{3x}{x^{2}-3x-18}\dif{x}\\[8pt]
    &= \frac{5}{2}x^{2} + 4x + \int\frac{2}{x-6} + \frac{1}{x+3}\\[8pt]
    &= \frac{5}{2}x^{2} + 4x + 2\ln(x-6) + \ln(x+3) + C\\[8pt]
\end{align*}


    \qpart
    \begin{align*}
        \stext{The matrix to represent reflection in the line \( y = -x \) is}
        R &= \begin{pmatrix}
            0 & -1\\
            -1 & 0
        \end{pmatrix}\\[8pt]
        \stext{To reflect in the line \( y = -x + 7 \), use translation \( h \) to the origin, apply \( R \), then translate back with \( h^{-1} \)}\\[8pt]
        \stext{Let \( h(x, y) = (x, y - 7) \), \( h^{-1}(x, y) = (x, y + 7) \)}\\[8pt]
        \stext{Apply the composite transformation \( f(x) = h^{-1}(R(h(x))) \)}\\[8pt]
        \stext{Step 1: Translate down by 7}
        h(x) &= \begin{pmatrix} x \\ y - 7 \end{pmatrix}\\[8pt]
        \stext{Step 2: Reflect in \( y = -x \)}
        R \cdot h(x) &= \begin{pmatrix}
            0 & -1 \\
            -1 & 0
        \end{pmatrix}
        \cdot
        \begin{pmatrix}
            x \\
            y - 7
        \end{pmatrix}
        =
        \begin{pmatrix}
            -(y - 7) \\
            -x
        \end{pmatrix}
        =
        \begin{pmatrix}
            7 - y \\
            -x
        \end{pmatrix}\\[8pt]
        \stext{Step 3: Translate up by 7}
        f(x) &= h^{-1}(R(h(x))) =
        \begin{pmatrix}
            7 - y \\
            -x + 7
        \end{pmatrix}\\[8pt]
        \stext{Therefore, the matrix form is:}
        B &= \begin{pmatrix}
            0 & -1 \\
            -1 & 0
        \end{pmatrix}, \quad
        b = \begin{pmatrix}
            7 \\
            7
        \end{pmatrix}
    \end{align*}

\end{question}

\begin{question}

\qpart

\[ \frac{5x^{3} - 11x^{2} - 99x - 72}{x^{2} - 3x - 18} \]

First we need to divide the polynomials using long division.

\polylongdiv{5x^{3}-11x^{2}-99x-72}{x^{2}-3x-18}

\begin{align*}
    \frac{5x^{3} - 11x^{2} - 99x - 72}{x^{2} - 3x - 18} &= 5x + 4 + \frac{3x}{x^{2} - 3x - 18}\\[8pt]
\stext{using partial fractions on the remainder}\\[8pt]
    \frac{3x}{x^{2} - 3x - 18} &= \frac{A}{x - 6} + \frac{B}{x + 3}\\[8pt]
\stext{Using augmented matrix to solve for \( A \) and \( B \)}\\[8pt]
    3x &= A(x + 3) + B(x - 6)\\[8pt]
    &= Ax + 3A + Bx - 6B\\[8pt]
    &= (A + B)x + (3A - 6B)\\[8pt]
\stext{Setting up the augmented matrix}\\[8pt]
    \begin{amatrix}{2}
        3 & -6 & 0\\
        1 & 1 & 3
    \end{amatrix}\\[8pt]
\stext{Subtract 3 times R2 from R1}\\[8pt]
    \begin{amatrix}{2}
        0 & -9 & -9\\
        1 & 1 & 3
    \end{amatrix}\\[8pt]
\stext{Divide R1 by -9}\\[8pt]
    \begin{amatrix}{2}
        0 & 1 & 1\\
        1 & 1 & 3
    \end{amatrix}\\[8pt]
\stext{Subtract R1 from R2}\\[8pt]
    \begin{amatrix}{2}
        0 & 1 & 1\\
        1 & 0 & 2
    \end{amatrix}\\[8pt]
\stext{Hence we can write}\\[8pt]
    \frac{3x}{x^{2} - 3x - 18} &= \frac{2}{x - 6} + \frac{1}{x + 3}\\[8pt]
\stext{Giving us the full expression}\\[8pt]
    \frac{5x^{3} - 11x^{2} - 99x - 72}{x^{2} - 3x - 18} &= 5x + 4 + \frac{2}{x - 6} + \frac{1}{x + 3}\\[8pt]
\end{align*}

\qpart
Maxima file

\qpart

\begin{align*}
\int \frac{5x^{3} - 11x^{2} - 99x - 72}{x^{2} - 3x - 18} \dd{x} &= \int \left(5x + 4 + \frac{2}{x - 6} + \frac{1}{x + 3}\right) \dd{x}\\[8pt]
&= \int 5x \dd{x} + \int 4 \dd{x} + \int \frac{2}{x - 6} \dd{x} + \int \frac{1}{x + 3} \dd{x}\\[8pt]
&= \frac{5}{2}x^{2} + 4x + 2\ln|x - 6| + \ln|x + 3| + C\\[8pt]
&= \frac{5}{2}x^{2} + 4x + 2\ln(x - 6) + \ln(x + 3) + C
\end{align*}

\end{question}

\end{document}