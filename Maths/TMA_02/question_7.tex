\documentclass[fleqn]{article}

%% Created with wxMaxima 25.01.0

\setlength{\parskip}{\medskipamount}
\setlength{\parindent}{0pt}
\usepackage{iftex}
\ifPDFTeX
  % PDFLaTeX or LaTeX 
  \usepackage[utf8]{inputenc}
  \usepackage[T1]{fontenc}
  \DeclareUnicodeCharacter{00B5}{\ensuremath{\mu}}
\else
  %  XeLaTeX or LuaLaTeX
  \usepackage{fontspec}
\fi
\usepackage{graphicx}
\usepackage{color}
\usepackage[leqno]{amsmath}
\usepackage{ifthen}
\newsavebox{\picturebox}
\newlength{\pictureboxwidth}
\newlength{\pictureboxheight}
\newcommand{\includeimage}[1]{
    \savebox{\picturebox}{\includegraphics{#1}}
    \settoheight{\pictureboxheight}{\usebox{\picturebox}}
    \settowidth{\pictureboxwidth}{\usebox{\picturebox}}
    \ifthenelse{\lengthtest{\pictureboxwidth > .95\linewidth}}
    {
        \includegraphics[width=.95\linewidth,height=.80\textheight,keepaspectratio]{#1}
    }
    {
        \ifthenelse{\lengthtest{\pictureboxheight>.80\textheight}}
        {
            \includegraphics[width=.95\linewidth,height=.80\textheight,keepaspectratio]{#1}
            
        }
        {
            \includegraphics{#1}
        }
    }
}
\newlength{\thislabelwidth}
\DeclareMathOperator{\abs}{abs}

\definecolor{labelcolor}{RGB}{100,0,0}

\begin{document}
Question 7
a)


\noindent
%%%%%%%%
%% INPUT:
\begin{minipage}[t]{4.000000em}\color{red}\bfseries
(\% i2)	
\end{minipage}
\begin{minipage}[t]{\textwidth}\color{blue}
f(x):=(3*x+15*x\^\ {2}-x\^\ {4})/(9*x\^\ {2}+1);
\end{minipage}
%%%% OUTPUT:
\[\displaystyle \tag{\% o2} 
\mathop{f}(x)\mathop{:=}\frac{3 x\mathop{+}15 {{x}^{{2}}}\mathop{-}{{x}^{{4}}}}{9 {{x}^{{2}}}\mathop{+}1}\mbox{}
\]
%%%%%%%%%%%%%%%%


\noindent
%%%%%%%%
%% INPUT:
\begin{minipage}[t]{4.000000em}\color{red}\bfseries
(\% i3)	
\end{minipage}
\begin{minipage}[t]{\textwidth}\color{blue}
f(1);
\end{minipage}
%%%% OUTPUT:
\[\displaystyle \tag{\% o3} 
\frac{17}{10}\mbox{}
\]
%%%%%%%%%%%%%%%%


\noindent
%%%%%%%%
%% INPUT:
\begin{minipage}[t]{4.000000em}\color{red}\bfseries
(\% i5)	
\end{minipage}
\begin{minipage}[t]{\textwidth}\color{blue}
plot2d([f(x)],\ [x,\ -5,\ 5],\ [y,\ -1,\ 2],\ \\
\ \ \ \ \ \ \ [plot\_format,\ xmaxima],\ \\
\ \ \ \ \ \ \ [gnuplot\_postamble,\ "set\ zeroaxis;"])\$
\end{minipage}
%%%% OUTPUT:
\[\displaystyle plot2d: expression evaluates to non-numeric value somewhere in plotting range.\mbox{}
\]
%%%%%%%%%%%%%%%%
\end{document}
