\documentclass[fleqn]{article}

%% Created with wxMaxima 25.01.0

\setlength{\parskip}{\medskipamount}
\setlength{\parindent}{0pt}
\usepackage{iftex}
\ifPDFTeX
  % PDFLaTeX or LaTeX 
  \usepackage[utf8]{inputenc}
  \usepackage[T1]{fontenc}
  \DeclareUnicodeCharacter{00B5}{\ensuremath{\mu}}
\else
  %  XeLaTeX or LuaLaTeX
  \usepackage{fontspec}
\fi
\usepackage{graphicx}
\usepackage{color}
\usepackage[leqno]{amsmath}
\usepackage{ifthen}
\newsavebox{\picturebox}
\newlength{\pictureboxwidth}
\newlength{\pictureboxheight}
\newcommand{\includeimage}[1]{
    \savebox{\picturebox}{\includegraphics{#1}}
    \settoheight{\pictureboxheight}{\usebox{\picturebox}}
    \settowidth{\pictureboxwidth}{\usebox{\picturebox}}
    \ifthenelse{\lengthtest{\pictureboxwidth > .95\linewidth}}
    {
        \includegraphics[width=.95\linewidth,height=.80\textheight,keepaspectratio]{#1}
    }
    {
        \ifthenelse{\lengthtest{\pictureboxheight>.80\textheight}}
        {
            \includegraphics[width=.95\linewidth,height=.80\textheight,keepaspectratio]{#1}
            
        }
        {
            \includegraphics{#1}
        }
    }
}
\newlength{\thislabelwidth}
\DeclareMathOperator{\abs}{abs}

\definecolor{labelcolor}{RGB}{100,0,0}

\begin{document}
Let A be the quantities of onions, carrots and garlic cloves needed to make vegetable , minestrone and French onion soup respectively.


\noindent
%%%%%%%%
%% INPUT:
\begin{minipage}[t]{4.000000em}\color{red}\bfseries
(\% i8)	
\end{minipage}
\begin{minipage}[t]{\textwidth}\color{blue}
A:\ matrix(\\
\ [2,2,8],\ \\
\ [3,1,0],\ \\
\ [2,3,4]\\
);
\end{minipage}
%%%% OUTPUT:
\[\displaystyle \tag{A} 
\begin{pmatrix}2 & 2 & 8\\
3 & 1 & 0\\
2 & 3 & 4\end{pmatrix}\mbox{}
\]
%%%%%%%%%%%%%%%%
Let x=vegtable, y=minestone and z=French onion.


\noindent
%%%%%%%%
%% INPUT:
\begin{minipage}[t]{4.000000em}\color{red}\bfseries
(\% i18)	
\end{minipage}
\begin{minipage}[t]{\textwidth}\color{blue}
B:\ matrix(\\
[x],\ \\
[y],\\
[z]\\
);
\end{minipage}
%%%% OUTPUT:
\[\displaystyle \tag{B} 
\begin{pmatrix}x\\
y\\
z\end{pmatrix}\mbox{}
\]
%%%%%%%%%%%%%%%%
Let C be the quatities of the ingredients in the store cupboard.


\noindent
%%%%%%%%
%% INPUT:
\begin{minipage}[t]{4.000000em}\color{red}\bfseries
(\% i39)	
\end{minipage}
\begin{minipage}[t]{\textwidth}\color{blue}
C:\ matrix(\\
\ \ \ \ [40],\ \\
\ \ \ \ [10],\ \\
\ \ \ \ [25]\\
);\\

\end{minipage}
%%%% OUTPUT:
\[\displaystyle \tag{C} 
\begin{pmatrix}40\\
10\\
25\end{pmatrix}\mbox{}
\]
%%%%%%%%%%%%%%%%
By using these matrices to represent the 3 simultaneous equation,
2x+2y+8z=40
3x+1y+0z=10
2x+3y+4z=25
The inverse of A is


\noindent
%%%%%%%%
%% INPUT:
\begin{minipage}[t]{4.000000em}\color{red}\bfseries
(\% i43)	
\end{minipage}
\begin{minipage}[t]{\textwidth}\color{blue}
A\_inv:invert(A);
\end{minipage}
%%%% OUTPUT:
\[\displaystyle \tag{A\_ inv} 
\begin{pmatrix}\frac{1}{10} & \frac{2}{5} & \mathop{-}\left( \frac{1}{5}\right) \\
\mathop{-}\left( \frac{3}{10}\right)  & \mathop{-}\left( \frac{1}{5}\right)  & \frac{3}{5}\\
\frac{7}{40} & \mathop{-}\left( \frac{1}{20}\right)  & \mathop{-}\left( \frac{1}{10}\right) \end{pmatrix}\mbox{}
\]
%%%%%%%%%%%%%%%%
By multiply both sides of the equations by the inverse of A


\noindent
%%%%%%%%
%% INPUT:
\begin{minipage}[t]{4.000000em}\color{red}\bfseries
(\% i42)	
\end{minipage}
\begin{minipage}[t]{\textwidth}\color{blue}
A\_inv.C;
\end{minipage}
%%%% OUTPUT:
\[\displaystyle \tag{\% o42} 
\begin{pmatrix}3\\
1\\
4\end{pmatrix}\mbox{}
\]
%%%%%%%%%%%%%%%%
So we can mke 3 portions of vegtable soup, 1 minestone soup and 4 French onion soup.
\end{document}
